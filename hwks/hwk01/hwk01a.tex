\documentclass[12pt]{article}
\usepackage[utf8]{inputenc}
\usepackage{geometry}
\usepackage{svg}
\usepackage{float}
\usepackage{caption}
\usepackage{amsmath,amsthm,amsfonts,amssymb,amscd}
\usepackage{fancyhdr}
\usepackage{titlesec}
\pagestyle{empty}
\titleformat*{\section}{\large\bfseries}

%
\geometry{letterpaper}                   		% ... or a4paper or a5paper or ... 
%\geometry{
% letter,
% total={170mm,240mm},
% left=20mm,
% top=30mm,
% }

%\date{}

%Bitte ausfüllen
\newcommand\course{Mec\'anica de Fluidos}
\newcommand\hwnumber{2015966}
\newcommand\Name{Alejandro Morales (Ph.D.)}
\newcommand\Neptun{Depto. Ing. Civil y Agr\'icola, UNAL}

%Matheinheiten
\newcommand\m{\:\textrm{m}}
\newcommand\M{\:\Big[\textrm{m}\Big]}
\newcommand\mm{\:\textrm{mm}}
\newcommand\MM{\:\Big[\textrm{mm}\Big]}
\newcommand\un{\underline}
\newcommand\s{\:\textrm{s}}
\newcommand\bS{\:\Big[\textrm{S}\Big]}
\newcommand\ms{\:\frac{\textrm{m}}{\textrm{s}}}
\newcommand\MS{\:\Big[\frac{\textrm{m}}{\textrm{s}}\Big]}
\newcommand\mss{\:\frac{\textrm{m}}{\textrm{s}^2}}
\newcommand\MSS{\:\Big[\frac{\textrm{m}}{\textrm{s}^2}\Big]}

%Bitte nicht einstellen
\pagestyle{fancyplain}
\headheight 35pt
\lhead{\Name\\\Neptun}
%\chead{\textbf{\Large Taller No.1a\\ asdfasdfsadf }}
\rhead{\course\ (\hwnumber) \\ 2022-II}
\lfoot{}
\cfoot{}
\rfoot{\small\thepage}
\headsep 1.5em

%\title{\vspace{-1cm}Taller 1a: Transformaci\'on de unidades y propiedades de los fluidos \vspace{-0.7cm}}
%\vspace{-3cm}
%\author{Alejandro Morales \\ \texttt{lmoralesm@unal.edu.co}}

\begin{document}
\begin{center}
\Large Taller No.1a\\ \emph{Transformaci\'on de unidades y propiedades de los fluidos}
\end{center}

%\maketitle

\section*{Ejercicio 1}\vspace{-0.3cm}
Las unidades de energ\'ia en el sistema ingl\'es de unidades est\'an dadas por ($lb\ pie$). Convierta una unidad de energ\'ia en este sistema al sistema internacional de unidades.

% White P1.12 
\section*{Ejercicio 2}\vspace{-0.3cm}
Para flujo permanente en una tuber\'ia circular a velocidad baja, la velocidad $u$ var\'ia como:
$$
u = B\frac{\Delta p}{\mu}(r^2_0 - r^2)
$$
donde $r$ es el radio de la tuber\'ia, $\mu$ es la viscosidad del fluido y $\Delta p$ es el cambio de presi\'on a lo largo de la tuber\'ia. ¿Cuales son las dimensiones de la constante $B$? 

% White P1.21
\section*{Ejercicio 3}\vspace{-0.3cm}
En 1908, Heinrich Blasius estudiante de Prandtl, propuso la siguiente f\'ormula para los esfuerzos cortantes ($\tau_w$) de un fluido viscoso:
$$
\tau_w = 0.332\rho^{1/2}\mu^{1/2}V^{3/2}x^{-1/2}
$$
en donde $V$ es la velocidad, $\rho$ es la densidad, $\mu$ es la viscosidad y $x$ es una distancia recorrida. Determine las dimensiones de la constante 0.332.


% White P1.9
\section*{Ejercicio 4}\vspace{-0.3cm}
Un contenedor c\'onico invertido, cuyo di\'ametro es 26 $pulg$ y altura 44 $pulg$ est\'a lleno con un liquido a 20 $^oC$. El peso del liquido son 5030 onzas. ¿Cual es la densidad del fluido en $kg/m^3$?


\section*{Ejercicio 5}\vspace{-0.3cm}
La viscosidad absoluta del mercurio es de 1.7 $cP$ y su gravedad espec\'ifica es 13.6 a temperatura y presi\'on atmosf\'erica estandard. Encuentre la viscosidad cinem\'atica en el sistema internacional y en el sistema Ingles.

% Street 1.11
\section*{Ejercicio 6}\vspace{-0.3cm}
Si un tanque con aceite pesa 1.5 $kN$, calcular el peso espec\'ifico, la densidad y la gravedad espec\'ifica del aceite. El barril contiene 159 litros y pesa 110 $N$ cuando esta vacio. 

\section*{Ejercicio 7}\vspace{-0.3cm}
Determine el peso y la gravedad especifica de un gal\'on de l\'iquido si este tiene una masa de 0.258 $slug$. Exprese todas las variables en el sistema Ingl\'es.

\section*{Ejercicio 8}\vspace{-0.3cm}
A presi\'on atmosf\'erica $P_{atm}=14.7\ psi$, un tanque contiene 120 $pie^3$ de agua que pesa 7488 $lb$. Determine:
\begin{itemize}
\item[a.] su densidad.
\item[b.] Si la presi\'on se eleva a 1470 $psi$, ¿cual es el valor de la densidad?
\end{itemize}

\section*{Ejercicio 9}\vspace{-0.3cm}
Si el modulo de elasticidad volum\'etrico del agua es de 300000 $psi$, ¿que presi\'on se requiere para reducir su volumen en 0.5\%?
	
% Street 1.29
\section*{Ejercicio 10}\vspace{-0.3cm}
Si el volumen de un l\'iquido es reducido en 0.035?\% mediante la aplicaci\'on de una presi\'on de 690 $kPa$ o 100 $psi$, ¿cual es el m\'odulo de elasticidad?


\end{document}

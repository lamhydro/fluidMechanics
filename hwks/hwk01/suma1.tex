\documentclass{article}
\usepackage[landscape]{geometry}
\usepackage{url}
\usepackage{multicol}
\usepackage{amsmath}
\usepackage{esint}
\usepackage{amsfonts}
\usepackage{tikz}
\usetikzlibrary{decorations.pathmorphing}
\usepackage{amsmath,amssymb}

\usepackage{colortbl}
\usepackage{xcolor}
\usepackage{mathtools}
\usepackage{amsmath,amssymb}
\usepackage{enumitem}
\makeatletter

\newcommand*\bigcdot{\mathpalette\bigcdot@{.5}}
\newcommand*\bigcdot@[2]{\mathbin{\vcenter{\hbox{\scalebox{#2}{$\m@th#1\bullet$}}}}}
\makeatother

\title{130 Cheat Sheet}
%\usepackage[brazilian]{babel}
\usepackage[utf8]{inputenc}

\advance\topmargin-.8in
\advance\textheight3in
\advance\textwidth3in
\advance\oddsidemargin-1.5in
\advance\evensidemargin-1.5in
\parindent0pt
\parskip2pt
\newcommand{\hr}{\centerline{\rule{3.5in}{1pt}}}
%\colorbox[HTML]{e4e4e4}{\makebox[\textwidth-2\fboxsep][l]{texto}
\begin{document}

\begin{center}{\huge{\textbf{Resume No.1: Propiedades de los fluidos}}}\\
\end{center}
\begin{multicols*}{2}

\tikzstyle{mybox} = [draw=black, fill=white, very thick,
    rectangle, rounded corners, inner sep=10pt, inner ysep=10pt]
\tikzstyle{fancytitle} =[fill=black, text=white, font=\bfseries]

%------------ Heating ---------------
\begin{tikzpicture}
\node [mybox] (box){%
    \begin{minipage}{0.45\textwidth}
\begin{center}
\begin{tabular}{ |c|c|c|c|c| } 
\hline
VARIABLE & DIMENSION & S.I & INGLES & CGS \\
\hline
Masa & M & kg & slug & gr \\ 
Longitud & L & m & pie & cm \\ 
Tiempo & T & s & s & s \\ 
Temperatura & $\theta$ & K$^o$ & R$^o$ & C$^o$ \\ 
Fuerza F & M L T$^{-2}$ & N = kg.m/s$^2$ & lb = slug.pie/s$^2$ & dina \\ 
%Presi\'on P & M L$^{-1}$ T$^{-2}$ & Pa=N/m$^2$ & lb/pie$^2$ & dina/cm$^2$  \\ 
%Densidad $\rho$ & M L$^{-3}$ Pa=N/m$^2$ & lb/pie$^2$ & dina/cm$^2$  \\ 
\hline
\end{tabular}
\end{center}
    \end{minipage}
};
%------------ Heating Header ---------------------
\node[fancytitle, right=10pt] at (box.north west) {Sistemas de unidades};
\end{tikzpicture}

%------------ Mixing ---------------
\begin{tikzpicture}
\node [mybox] (box){%
    \begin{minipage}{0.45\textwidth}
\begin{center}
\small
\begin{tabular}{ c c }
 1 m = & 3.28 pies \\
 1 pie = & 0.3048 m = 30.48 cm \\
 1 pulg = & 2.54 cm \\
 1 pie = & 12 pulg \\
 1 gal = & 3.785 lt \\
 1 kg = &  2.2 lb \\
 1 kg = &  9.8 N \\
 1 bar = & 10$^5$ Pa \\
 1 mbar = & 100 Pa \\
 1 psi = & 1 lb/pulg$^2$ \\
 1 hp = & 550 lb.pie/s \\
 1 N = &  10$^5$ dinas \\
 1 Poise = & gr/(cm.s) \\
 1 Stoke = & cm$^2$/s \\
 1 Pa = & 1 N/m$^2$ \\
 1 Joule = & 1 N.m \\
 1 Watt = & 1 Joule/s \\
 1 slug = & 32.2 lb \\
 1 lb = & 16 onzas \\
 1 $^o$K = & 273 + $^o$C \\
 1 $^o$R = & $\frac{9}{5}$ $^o$K \\
\end{tabular}
\end{center}
    \end{minipage}
};
%------------ Mixing Header ---------------------
\node[fancytitle, right=10pt] at (box.north west) {Transformaci\'on de unidades};
\end{tikzpicture}

%------------ Inner Product Spaces ---------------
\begin{tikzpicture}
\node [mybox] (box){%
    \begin{minipage}{0.45\textwidth}

\begin{center}
\begin{tabular}{ c c }
\vspace{0.3cm}
Peso & $W = M.g$ \\ \vspace{0.2cm}
Presion & $P = \frac{F_p}{A}$ \\ \vspace{0.2cm}
Esfuerzo de corte & $\tau = \frac{F_f}{A}$ \\ \vspace{0.2cm}
Densidad & $\rho = \frac{M}{V}$ \\ \vspace{0.2cm}
Volmen espec\'ifico & $Vs = \frac{V}{M} = 1/\rho$ \\ \vspace{0.2cm}
Peso espec\'ifico & $\gamma = \frac{W}{V} = \rho . g$ \\ \vspace{0.2cm}
Densidad relativa o gravedad espec\'ifica & $S = \frac{\gamma}{\gamma_{H_2 O}}$ \\ \vspace{0.2cm}
Modulo de elasticidad volum\'etrica & $Ev =-V \frac{dP}{dV} = \rho \frac{dP}{d\rho} $ \\ \vspace{0.2cm}
Viscosidad cinem\'atica o relativa & $\nu =\frac{\mu}{\rho} $ 
\end{tabular}
\end{center}
    \end{minipage}
};
%------------ Inner Product Space Header ---------------------
\node[fancytitle, right=10pt] at (box.north west) {Propiedades de los fluidos};
\end{tikzpicture}

%------------ Gram-Schmidt Content ---------------
\begin{tikzpicture}
\node [mybox] (box){%
    \begin{minipage}{0.45\textwidth}
\begin{center}
\begin{tabular}{ |c|c|c|c| } 
\hline
VARIABLE & S.I & INGLES  \\
\hline
$g$ & 9.8 $\frac{m}{s^2}$ & 32.2 $\frac{pie}{s^2}$ \\ 
$\rho^*$ & 1000 $\frac{kg}{m^3}$ & 1.94 $\frac{slug}{pie^3}$ \\ 
\hline
\end{tabular}
\end{center}
 * En condiciones estandard de presi\'on y temperatura.
    \end{minipage}
};
%------------ Gram-Schmidt Header ---------------------
\node[fancytitle, right=10pt] at (box.north west) {Algunas contantes};
\end{tikzpicture}
%------------ Variation of Parameters Content ---------------------
\begin{tikzpicture}
\node [mybox] (box){%
    \begin{minipage}{0.45\textwidth}
    $$
    \tau = \mu \frac{du}{dy}
    $$
donde $\tau$ es el esfuerzo de corte, $u$ es la velocidad en funci\'on de $y$ y $\mu$ es la viscosidad din\'amica. Frecuentemente en los problemas se asume una distribucion lineal de velocidades porl lo que:
    $$
    \tau = \mu \frac{V}{h}
    $$
    donde $h$ es el espesor de la capa de fluido.
    \end{minipage}
};
%------------ Variation of Parameters Header ---------------------
\node[fancytitle, right=10pt] at (box.north west) {Ley de viscosidad de Newton};
\end{tikzpicture}

\begin{tikzpicture}
\node [mybox] (box){%
    \begin{minipage}{0.45\textwidth}
    \small
    \begin{enumerate}
    \item En caso de no existir, hacer dibujo con todos los datos del problema. Dibujar las fuerzas que producen el movimiento ($F$) y las fuerzas de corte ($F_c =\tau\ A$), donde $\tau = \mu \frac{du}{dy}$.
    \item Hacer el balance de fuerzas actuantes en la direcci\'on del movimiento. Ej. $F=F_c$ o $T/d=F=F_c$, donde $T$ es torque o par y $d$ es el brazo del torque.
    \item Asegurarse de que las unidades de todos los datos est\'en en el mismo sistema de unidades (S.I. o Ingles)
    \item Resolver el problema. Note que cuando las superficies sobre las cuales act\'uan las fuerzas cortantes del fluido cambian e.g. c\'onos, esferas, etc, el balance de fuerzas debe hacerse en t\'erminos de un diferencial (para ejercicios con conos $dT=r.dF$), ya que las fuerzas cambian y es necesario integrar.  
    \end{enumerate}
    \end{minipage}
};
%------------ Variation of Parameters Header ---------------------
\node[fancytitle, right=10pt] at (box.north west) {Pasos para resolver problemas ley de viscosidad de Newton};
\end{tikzpicture}

\begin{tikzpicture}
\node [mybox] (box){%
    \begin{minipage}{0.45\textwidth}
    \small
    La ley de los gases perfectos es:
     $$
    \frac{P}{\rho} = P V_s = RT
    $$
donde $P$ es la presi\'on absoluta del gas, $\rho=1/V_s$ es la densidad o el volumen espec\'ifico, $T$ es la temperatura absoluta, y $R$ es la constante del gas $R=\frac{8312}{M}$ ($\frac{N.m}{kg. ^oK}$) en S.I. y  $R=\frac{1545}{M}$ ($\frac{lb.pie}{slug. ^oR}$) en sistema Ingl\'es, donde $M$ es la masa molecular del gas. 
    \begin{itemize}
    \item Proceso isoentr\'opico: $P_1 V_{s1}^k = P_2 V_{s2}^k$
    \item Proceso isot\'ermico: $P_1 V_{s1} = P_2 V_{s2}$. $T$ no cambia
    \end{itemize}
    $k$ es el coeficiente de expansi\'on adiab\'atica. $Ev=kP$ para gases perfectos. Celeridad del sonido en un gas es $c=\sqrt{kRT}$.
    \end{minipage}
};
%------------ Variation of Parameters Header ---------------------
\node[fancytitle, right=10pt] at (box.north west) {Gases perfectos};
\end{tikzpicture}

\end{multicols*}
\end{document}

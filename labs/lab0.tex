\documentclass[12pt, a4paper]{exam}
\usepackage{graphicx}
\usepackage[a4paper, total={6in, 8in}]{geometry}
\usepackage[normalem]{ulem}
\usepackage{amsmath}
\renewcommand\ULthickness{1.0pt}   
\setlength\ULdepth{1.3ex}

\begin{document}


	\noindent
	\begin{minipage}[l]{0.1\textwidth}
		\noindent
		\includegraphics[width=2.8\textwidth]{ESCUDO.png}
	\end{minipage}
\hfill
\begin{minipage}[c]{0.8\textwidth}
	\begin{center}
		{\large  Departamento de Ingeniería civil y Agrícola\par
		\large	Facultad de Ingeniería	\par
	% \large \textbf{Taller propiedades de los fluidos}	\par
    \large \textbf{Lab0 (Demostrativo) \\ Aparatos de medici\'on de presi\'on}	\par
} %%%%% NOMBRE DEL PROFESOR 
	\end{center}
\end{minipage}
\par
\vspace{0.2in}
\noindent
    \uline{Mecánica de fluidos [2015966]	\hfill 2022-II	}
\par 
\vspace{0.15in}
\noindent

%%%%%%%%%%%%%%%%%%%%%%%%%%%%%
\section{Normas del Laboratorio}
Es necesario vestir bata o overol para la realizaci\'on del laboratorio. Este atuendo puede ser de cualquier color o fabricante. Traer calculadora, l\'apiz y papel.

\section{Objetivo}
\begin{itemize}
\item Familiarizarnos con los aparatos para la medici\'on de presiones en diferentes tipos de sistemas a presi\'on con fluidos compresibles e incompresibles. 
\item Calcular diferencias de presi\'on a lo largo del sistema debido a accesorios, cambios de secci\'on y rugosidad del material de la tuber\'ia.
\item Diferenciar los tipos de aparatos de medici\'on de presi\'on, sus ventajas y desventajas.
\item Observar el fen\'omeno de presi\'on de vacio o presi\'on negativa en sistemas de bombeo. 
\end{itemize}
\section{Metodolog\'ia}
Esta practica es demostrativa y se hara en cuatro diferentes experimentos:
\begin{enumerate}
\item Sistema de circulaci\'on de aire a trav\'ez de tres tuber\'ias en paralelo de diferente di\'ametro. Man\'ometros conectados a lo largo de la tuber\'ia y tubo Pitot para determinar el caudal.
\item Red de tuber\'ias para la circulaci\'on de aire con variaci\'on de temperatura. El circuito consta de man\'ometros de diferentes tipos como: diferenciales, Burdon y transductores, conectados en diferentes puntos de la red. La red consta de v\'alvulas que se pueden operar manualmente y otras autom\'aticamente desde un panel.
\item Tuber\'ia en la que circula aceite a presi\'on. El caudal es controlado por un panel que permite ademas medir presi\'on en diferentes puntos. Man\'ometros son conectados a lo largo de la tuber\'ia para medir manualmente presi\'on en diferentes puntos. Con el man\'ometro cerca de la bomba es posible medir presiones negativas en la bomba dependiendo del caudal.   
\item Tuber\'ias en paralelo que transportan agua a presi\'on. Cada tuber\'ia es de diferente material y tienen transiciones bruscas y suaves para cambio de di\'ametro. Cada tuber\'ia tiene una serie man\'ometros conectados para determinar los cambios de presi\'on debido a la rugosidad o a los accesorios. Un vertedero de cresta delgada es usado para estimar el caudal que circula por cada tuber\'ia. 
\item Flujo a trav\'ez de una estructura. Este experimento nos permite observar y escuchar el fen\'omeno de presi\'on de cavitaci\'on producido por presiones de vacio en cercan\'ias a la estructura debido a un alto caudal que pasa a trav\'ez de la estructura.
\end{enumerate}
\section{Resultados esperados}
Para cada experimento, se propone determinar lo siguiente:
\begin{itemize}
\item \textbf{Experimento 1:} Calcular, usando los mismos man\'ometros, las diferencias de presi\'on en las tres tuber\'ias. 
\item \textbf{Experimento 2:} Determinar la propiedades del gas ($\rho$, $T$). Estimar diferencias de presi\'on para diferentes valores de $T$ en uno de los circuitos. Leer diferentes tipos de man\'ometros para corroborar resultados.  
\item \textbf{Experimento 3:} Determinar la propiedades del aceite ($\rho$ y $\mu$). Calcular el caudal del aceite que pasa por la tuber\'ia para un intervalo de tiempo. Determinar la diferencia de presiones antes y despu\'es de la bomba.
\item \textbf{Experimento 4:} Determinar las diferencias de presi\'on antes y despu\'es de accesorios que reducen o ampl\'ian el di\'ametro de la tuber\'ia. Analisar dichas diferencias cuando la transici\'on es brusca o gradual.
\item \textbf{Experimento 5:} Observar el vacio que se crea al incrementar el caudal que pasa a trav\'es de la estructura. Observar las lineas de corriente. Escuchar los estallidos r\'apidos de las burbujas de agua en el fen\'omeno de la cavitaci\'on. 
\end{itemize}

\section{Listado de instrumentos usados}
\begin{enumerate}
\item Medici\'on de presi\'on:
\begin{enumerate}
\item Man\'ometro de agua o de mercurio. El segundo sirve para medir presiones mayores.
\item Man\'ometro Burdon (Mediciones en PSI o columna de Hg)
\item Transductor de presi\'on que reporta los valores en un panel de control.
\end{enumerate}
\item Medici\'on de caudal:
\begin{enumerate}
\item Tubo Pitot: Es un peque\~no tubo en L que se ubica en sentido opuesto al flujo utilizado para medir la velocidad con base en una diferencia de presiones utilizando la ecuaci\'on de Bernoulli. La sumatorio de los valores de los velocidades por el \'area respectiva es igual al caudal de flujo. 
\item Vertedero de cresta delgada: Vertedero triangular que sirve para medir la altura sobre la cresta del vertedero para luego, mediante una ecuaci\'on de calibraci\'on, calcular el caudal.
\item Tubo Venturi: Es un tubo con una reducci\'on r\'apida de la secci\'on transversal que es seguido de una ampliaci\'on gradual de la secci\'on. En la reducci\'on de la linea de gradiente hidr\'aulica cae sustancialmente debido a una disminuci\'on notable de la presi\'on. Utilizando medidas de la presi\'on antes y en la reducci\'on, la ecuaci\'on de Bernoulli y el principio de conservaci\'on de masa es posible calcular el caudal.
\item Medidores de caudal: Instrumentos digitales para medir el caudal.
\end{enumerate}
\end{enumerate}

\end{document}




\documentclass [xcolor=svgnames, t] {beamer} 
\usepackage[utf8]{inputenc}
\usepackage{booktabs, comment} 
\usepackage[absolute, overlay]{textpos} 
\usepackage{pgfpages}
\usepackage[font=footnotesize]{caption}
\useoutertheme{infolines} 

\AtBeginSection[]{
  \begin{frame}
  \vfill
  \centering
  \begin{beamercolorbox}[sep=8pt,center,shadow=true,rounded=true]{title}
    \usebeamerfont{title}\insertsectionhead\par%
  \end{beamercolorbox}
  \vfill
  \end{frame}
}


%\definecolor{brownbrown}{RGB}{56, 28, 0}
%\definecolor{brownred}{RGB}{228, 0, 43}

%\setbeamercolor{title in head/foot}{bg=brownred, fg=brownbrown}
%\setbeamercolor{author in head/foot}{bg=myuniversity}
\setbeamertemplate{page number in head/foot}{}
\usepackage{csquotes}


\usepackage{amsmath}
\usepackage[makeroom]{cancel}


\usepackage{textpos}

\usepackage{tikz}

\usepackage{media9} 

\usetheme{Madrid}
%\definecolor{myuniversity}{RGB}{56, 28, 0}
%\usecolortheme[named=myuniversity]{structure}
\usepackage{tikz}



\title[Propie. Fluidos]{Clase No.2: Propiedades de los fluidos}
\subtitle{Sistemas de unidades, cifras significativas y transformaci\'on de unidades}
\institute[]{Departamento de Ingenier\'ia Civil y Agr\'icola\\ Facultad de Ingenier\'ia  \\Universidad Nacional de Colombia - Sede Bogot\'a}
\titlegraphic{\includegraphics[height=2.0cm]{escudoUnal.png}}
\author[LAM]{Luis Alejandro Morales \\ \href{https://lamhydro.github.io}{https://lamhydro.github.io}}


%\institute[]{Department of Earth, Environmental, and Planetary Sciences  \\Brown University}
\date{\today}


\addtobeamertemplate{navigation symbols}{}{%
    \usebeamerfont{footline}%
    \usebeamercolor[fg]{footline}%
    \hspace{1em}%
    \insertframenumber/\inserttotalframenumber
}

\begin{document}
\begin{frame}
\maketitle
\end{frame}


%%%%%%%%%%%%%%%%%%%%%%%%%%%%
\logo{\vspace{-0.2cm}\includegraphics[height=0.8cm]{escudoUnal.png}~%
}
%%%%%%%%%%%%%%%%%%%%%%%%%%



\begin{frame}
\frametitle{Table of Contents}
\tableofcontents
\end{frame}


\section{Sistemas de unidades}

\begin{frame}{Unidades primarias}
\centering
\includegraphics[width=12cm]{primu}
\end{frame}

\begin{frame}{Unidades secundarias}
\centering
\includegraphics[width=12cm]{secdim}
\end{frame}


\section{Transformaci\'on de unidades}
\begin{frame}{Transformaci\'on de unidades}
\centering
\includegraphics[width=11cm]{unitsDuarte}
\end{frame}



\begin{frame}{Transformaci\'on de unidades}
\vspace{-1.0cm}
\begin{columns}
\column{0.3\textwidth}
\begin{exampleblock}{}
El SI esta basado en relaciones decimales entre unidades. Los prefijos usados para expresar los multiples de varios unidades estan en la Tabla~\ref{t1}. Estos prefijos son estandards para todas las unidades.
\end{exampleblock}{}

\column{0.7\textwidth}
\scriptsize
\begin{table}[h!]
\centering
\begin{tabular}{c c}
 \hline
 Multiplo & Prefijo \\ [0.5ex]
 \hline\hline
 10$^{24}$ & yotta, Y \\
 10$^{21}$ & zetta, Z \\
 10$^{18}$ & exa, E \\
 10$^{15}$ & peta, P \\
 10$^{12}$ & tera, T \\
 10$^{9}$ & giga, G \\
 10$^{6}$ & mega, M \\
 10$^{3}$ & kilo, k \\
 10$^{2}$ & hecto, h \\
 10$^{1}$ & deka, da \\
 10$^{-1}$ & deci, d \\
 10$^{-2}$ & centi, c \\
 10$^{-3}$ & mili, m \\
 10$^{-6}$ & micro, $\mu$ \\
 10$^{-9}$ & nano, n \\
 10$^{-12}$ & pico, p \\
 10$^{-15}$ & femto, f \\
 10$^{-18}$ & atto, a \\
 10$^{-21}$ & zepto, z \\
 10$^{-24}$ & yocto, y \\ [1ex]
  \hline
\end{tabular}
%\caption{Prefijos estandard en el SI de unidades.}
\label{t1}
\end{table}
\end{columns}
\end{frame}

\section{Dimensiones homogeneas}
\begin{frame}{Dimensiones homogeneas}
\begin{exampleblock}{}
En ingenieria y ciencias, todas las ecuaciones deben ser dimensionalmente homogeneas, esto quiere decir que las cantidades que se suman o restan en la equacion tienen las mismas dimensiones. Por ejemplo, en la ecuaci\'on de Bernoulli para flujo incompresible:
$$
p + \frac{1}{2} \rho V^2 + \rho g Z = constante
$$
todos los terminos de la ecuaci\'on tienen dimensiones de presi\'on $[ML^{-1}T^{-2}]$.
\end{exampleblock}{}
\end{frame}


%\begin{frame}{Informaci\'on}
%\begin{exampleblock}{}
%\begin{itemize}
%\item \alert{Oficina}: Edificio Laboratorio de Hidr\'aulica, oficina 304 
%\item \alert{Email}: lmoralesm@unal.edu.co
%\end{itemize}
%\end{exampleblock}
%\end{frame}
%
%\begin{frame}{La asignatura}
%\begin{itemize}
%\item Identificaci\'on de la asignatura: \alert{2015966}
%\item Duraci\'on clase presencial: \alert{5 hrs/semana}
%\item No. de semanas: \alert{16}
%\item No. de cr\'editos: \alert{4}
%\item ¿Esta asignatura es validable?: \alert{No}
%\item ¿Asignatura de libre elecci\'on?: \alert{No}
%\item Planes de estudio a los que se asocia la asignatura
%\begin{itemize}
%\item \alert{2540 Ingenier\'ia Agr\'icola Componente C}
%\item \alert{2541 Ingenier\'ia Civil Componente C}
%\end{itemize}
%\item Prerequisitos
%\begin{itemize}
%\item \alert{C\'alculo en varias variables}
%\item \alert{Ecuaciones diferenciales}
%\item \alert{Est\'atica}
%\end{itemize}
%\end{itemize}
%\end{frame}
%
%\begin{frame}{P\'agina de la asignatura}
%\begin{block}{P\'agina de la asignatura}
%\href{https://lamhydro.github.io/fluidMechanics/}{https://lamhydro.github.io/fluidMechanics/}
%\end{block}
%\end{frame}
%
%\section{Horario de clases}
%\begin{frame}{Horario de clases}
%\centering
%\includegraphics[width=9cm]{horarioClases}
%\end{frame}
%
%\section{Descripci\'on}
%\begin{frame}{Descripci\'on}
%\vspace{-0.8cm} 
%\begin{columns}
%\column{0.6\textwidth}
%\begin{exampleblock}{}
%\small
%La asignatura mecánica de fluidos se centra en el estudio y análisis de las propiedades físicas más relevantes de los fluidos a partir de los aspectos fundamentales de la física, el cálculo y las ecuaciones diferenciales. Con ello el curso se centra en el estudio y aplicación de \alert{cuatro principios: de Pascal, de conservación de la masa, de conservación de la energía, momentum lineal y angular}, principios que se aplican a diferentes condiciones de contorno a los cuales están sometidos los fluidos. Finalmente se hace una introducción al análisis dimensional y semejanza hidráulica.
%\end{exampleblock}
%\column{0.4\textwidth}
%\begin{center}
%\includegraphics[width=\textwidth]{fm}
%\includegraphics[width=\textwidth]{fm1}
%\end{center}
%\end{columns}
%\end{frame}
%
%
%\section{Conceptos previos necesarios}
%\begin{frame}{Conceptos previos necesarios}
%\begin{exampleblock}{}
%El curso requiere el uso de herramientas matemáticas para la solución de \alert{ecuaciones diferenciales} y el entendimiento básico de los teoremas fundamentales del \alert{cálculo multivariado}, así como los conocimientos básicos de Estática. Es deseable que los estudiantes tengan conocimientos básicos en programación de computadores y el manejo de herramientas de métodos numéricos.
%\end{exampleblock}
%\end{frame}
%
%\section{Objetivo}
%\begin{frame}{Objetivo}
%\begin{columns}
%\column{0.6\textwidth}
%\begin{exampleblock}{}
%Dar a conocer las \alert{leyes físicas fundamentales} que tienen que ver con el comportamiento de los fluidos y sus aplicaciones a \alert{problemas típicos de la Ingeniería}. Al finalizar el curso, el estudiante debe estar en capacidad de aplicar las ecuaciones básicas de mecánica de fluidos.
%\end{exampleblock}
%\column{0.4\textwidth}
%\begin{center}
%\includegraphics[width=\textwidth]{fmo1}
%\end{center}
%\end{columns}
%\end{frame}
%
%\section{Contenido general de la materia}
%
%\begin{frame}{1. Propiedades de los fluidos}
%\begin{exampleblock}{}
%\begin{itemize}
%\item Sistemas de unidades
%\item Transformación de unidades
%\item Definición de presión y esfuerzo de corte
%\item Ley de viscosidad de Newton
%\item Tipos de fluidos y tipos de flujo
%\item Propiedades de los fluidos
%\item Ecuación de estado de los gases y gases perfectos
%\end{itemize}
%\end{exampleblock}
%\end{frame}
%
%\begin{frame}{2. Est\'atica de los fluidos}
%\vspace{-0.9cm}
%\begin{columns}
%\column{0.6\textwidth}
%\begin{exampleblock}{}
%\begin{itemize}
%\small
%\item Fluidos en reposo, escala de medida de la presión y de temperatura
%\item Ecuación fundamental de la estática de los fluidos, fluido incompresible, fluido compresible, principio de Pascal
%\item Aparatos medidores de presión
%\item Fuerzas sobre cuerpos sumergidos: superficies planas, superficies curvas, principios de flotación
%\item Equilibrio relativo de fluidos en movimiento
%\item Aceleración lineal uniforme, rotación uniforme alrededor de un eje vertical
%\item \alert{Laboratorios: Aparatos medidores de presión, Ensayos de flotación, Propiedades de los fluidos (banco de estática), Aparato de Reynolds}
%\end{itemize}
%\end{exampleblock}
%\column{0.4\textwidth}
%\begin{center}
%\includegraphics[width=\textwidth]{fme}
%\includegraphics[width=\textwidth]{fme1}
%\end{center}
%\end{columns}
%\end{frame}
%
%\begin{frame}{3. Cinem\'atica de los fluidos}
%\vspace{-0.9cm}
%\begin{columns}
%\column{0.6\textwidth}
%\begin{exampleblock}{}
%\begin{itemize}
%\item Generalidades, propiedades cinemáticas del flujo
%\item Métodos para describir el movimiento de un fluido: método de Lagrange y método de Euler
%\item Flujo volumétrico y flujo másico, línea de corriente y clasificación de los flujos
%\item Teorema de Transporte de Reynolds, ecuación de continuidad para un volumen de control, continuidad en un punto
%\item Flujo potencial y función de corriente, relación entre flujo potencial y función de corriente
%\item \alert{Laboratorio: Hele-Shaw}
%\end{itemize}
%\end{exampleblock}
%\column{0.4\textwidth}
%\begin{center}
%\includegraphics[width=\textwidth]{fmk}
%\end{center}
%\end{columns}
%\end{frame}
%
%\begin{frame}{4. Din\'amica de los fluidos}
%\vspace{-1.2cm}
%\begin{columns}
%\column{0.6\textwidth}
%\begin{exampleblock}{}
%\begin{itemize}
%\footnotesize
%\item Ecuación de energía para flujos incompresibles y significado físico
%\item Concepto de línea de energía, línea de gradiente hidráulico y potencia hidráulica
%\item Aplicaciones de la ecuación de energía a sistemas de conducción: tuberías, canales, sistemas de bombeo, sistemas de generación hidroeléctrica
%\item Medidores de caudal: orificios, tubo Pitot, tubo Venturi, medidor de codo
%\item Cantidad de movimiento: cantidad de movimiento lineal aplicado a: estructuras hidráulicas como compuertas, vertederos, salto hidráulico, accesorios en tuberías, álabes fijos y móviles. 
%\item Cantidad de movimiento angular.
%\scriptsize
%\item \alert{Laboratorios: Línea de energía y línea de gradiente hidráulico, Flujo compresible, Tubo Pitot, Aparatos medidores de caudal: orificio, tubo Venturi.}
%\end{itemize}
%\end{exampleblock}
%\column{0.4\textwidth}
%\begin{center}
%\includegraphics[width=\textwidth]{fmd}
%\includegraphics[width=\textwidth]{fmd1}
%\end{center}
%\end{columns}
%
%\end{frame}
%
%\begin{frame}{5. Analisis dimensional}
%\begin{columns}
%\column{0.5\textwidth}
%\begin{exampleblock}{}
%\begin{itemize}
%\item Ecuación dimensionalmente homogénea
%\item Parámetros adimensionales relevantes en mecánica de fluidos
%\item Obtención de ecuaciones y teorema pi de Buckingham
%\item Determinación de los grupos adimensionales y leyes de semejanza
%\item Clasificación de los modelos físicos y aplicaciones
%\end{itemize}
%\end{exampleblock}
%\column{0.5\textwidth}
%\begin{center}
%\includegraphics[width=\textwidth]{dimana}
%\end{center}
%\end{columns}
%\end{frame}
%
%\section{Evaluaci\'on del curso}
%\begin{frame}{Evaluaci\'on del curso}
%\centering
%\includegraphics[width=9cm]{eval}
%\end{frame}
%
%\section{Referencias}
%\begin{frame}{Referencias}
%\begin{block}{Recomendadas}
%\begin{enumerate}
%\item \textbf{Çengel, Y. A., \& Cimbala, J. M. (2010). Fluid Mechanics: Fundamentals and Applications, McGraw-Hill Higher Education.}
%\item \textbf{Duarte, C. A. (2017). Mecánica de fluidos e hidráulica, Universidad Nacional de Colombia, Facultad de Ingeniería,Departamento de Ingeniería Civil y Agrícola.}
%\item \textbf{White, F. (2015). Fluid Mechanics, McGraw-Hill Higher Education.}
%\end{enumerate}
%\end{block}
%\end{frame}
%
%\begin{frame}{Referencias}
%\begin{block}{Otros}
%\begin{enumerate}
%\small
%\item Franzini, J. B., \& Finnemore, E. J. (1997). Fluid Mechanics with Engineering Applications: McGraw-Hill.
%\item Liu, C. (2013). Schaum’s Outline of Fluid Mechanics and Hydraulics, 4th Edition: McGraw-Hill Education.
%\item Mott, R. L. (2006). Applied Fluid Mechanics: Pearson Prentice Hall.
%\item Potter, M. C., Wiggert, D. C., \& Ramadan, B. H. (2016). Mechanics of Fluids, SI Edition: Cengage Learning.
%\item Pritchard, P. J. (2010). Fox and McDonald's Introduction to Fluid Mechanics, 8th Edition: John Wiley \& Sons.
%\item Roberson, J. A., \& Crowe, C. T. (1999). Engineering Fluid Mechanics: Jaico Publishing House.
%\item Shames, I. H. (1992). Mechanics of Fluids: McGraw-Hill.
%\item Sotelo, G. (1974). Hidráulica general: fundamentos: Limusa.
%\item Street, R. L., Watters, G. Z., \& Vennard, J. K. (1995). Elementary Fluid Mechanics: Wiley.
%\item Streeter, V. L., \& Wylie, E. B. (1979). Fluid mechanics: McGraw-Hill.
%\end{enumerate}
%\end{block}
%\end{frame}

%\begin{frame} [allowframebreaks]\frametitle{References}
               
%        \bibliographystyle{apalike}
%        \bibliography{bibfile}
%\end{frame}

\end{document}


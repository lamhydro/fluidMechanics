\documentclass [xcolor=svgnames, t] {beamer} 
\usepackage[utf8]{inputenc}
\usepackage{booktabs, comment} 
\usepackage[absolute, overlay]{textpos} 
\usepackage{pgfpages}
\usepackage[font=footnotesize]{caption}
\useoutertheme{infolines} 

\AtBeginSection[]{
  \begin{frame}
  \vfill
  \centering
  \begin{beamercolorbox}[sep=8pt,center,shadow=true,rounded=true]{title}
    \usebeamerfont{title}\insertsectionhead\par%
  \end{beamercolorbox}
  \vfill
  \end{frame}
}


%\definecolor{brownbrown}{RGB}{56, 28, 0}
%\definecolor{brownred}{RGB}{228, 0, 43}

%\setbeamercolor{title in head/foot}{bg=brownred, fg=brownbrown}
%\setbeamercolor{author in head/foot}{bg=myuniversity}
\setbeamertemplate{page number in head/foot}{}
\usepackage{csquotes}


\usepackage{amsmath}
\usepackage[makeroom]{cancel}


\usepackage{textpos}

\usepackage{tikz}

\usepackage{media9} 

\usetheme{Madrid}
%\definecolor{myuniversity}{RGB}{56, 28, 0}
%\usecolortheme[named=myuniversity]{structure}
\usepackage{tikz}



\title[Propie. Fluidos]{Clase No.3: Propiedades de los fluidos}
\subtitle{Transformaci\'on de unidades}
\institute[]{Departamento de Ingenier\'ia Civil y Agr\'icola\\ Facultad de Ingenier\'ia  \\Universidad Nacional de Colombia - Sede Bogot\'a}
\titlegraphic{\includegraphics[height=2.0cm]{escudoUnal.png}}
\author[LAM]{Luis Alejandro Morales \\ \href{https://lamhydro.github.io}{https://lamhydro.github.io}}


%\institute[]{Department of Earth, Environmental, and Planetary Sciences  \\Brown University}
\date{\today}


\addtobeamertemplate{navigation symbols}{}{%
    \usebeamerfont{footline}%
    \usebeamercolor[fg]{footline}%
    \hspace{1em}%
    \insertframenumber/\inserttotalframenumber
}

\begin{document}
\begin{frame}
\maketitle
\end{frame}


%%%%%%%%%%%%%%%%%%%%%%%%%%%%
\logo{\vspace{-0.2cm}\includegraphics[height=0.8cm]{escudoUnal.png}~%
}
%%%%%%%%%%%%%%%%%%%%%%%%%%



\begin{frame}
\frametitle{Table of Contents}
\tableofcontents
\end{frame}


\section{Transformaci\'on de unidades}
\begin{frame}{Transformaci\'on de unidades}
\vspace{-1.0cm}
\begin{columns}
\column{0.3\textwidth}
\begin{exampleblock}{}
El SI esta basado en relaciones decimales entre unidades. Los prefijos usados para expresar los multiples de varios unidades estan en la Tabla~\ref{t1}. Estos prefijos son estandards para todas las unidades.
\end{exampleblock}{}

\column{0.7\textwidth}
\scriptsize
\begin{table}[h!]
\centering
\begin{tabular}{c c}
 \hline
 Multiplo & Prefijo \\ [0.5ex]
 \hline\hline
 10$^{24}$ & yotta, Y \\
 10$^{21}$ & zetta, Z \\
 10$^{18}$ & exa, E \\
 10$^{15}$ & peta, P \\
 10$^{12}$ & tera, T \\
 10$^{9}$ & giga, G \\
 10$^{6}$ & mega, M \\
 10$^{3}$ & kilo, k \\
 10$^{2}$ & hecto, h \\
 10$^{1}$ & deka, da \\
 10$^{-1}$ & deci, d \\
 10$^{-2}$ & centi, c \\
 10$^{-3}$ & mili, m \\
 10$^{-6}$ & micro, $\mu$ \\
 10$^{-9}$ & nano, n \\
 10$^{-12}$ & pico, p \\
 10$^{-15}$ & femto, f \\
 10$^{-18}$ & atto, a \\
 10$^{-21}$ & zepto, z \\
 10$^{-24}$ & yocto, y \\ [1ex]
  \hline
\end{tabular}
%\caption{Prefijos estandard en el SI de unidades.}
\label{t1}
\end{table}
\end{columns}
\end{frame}

\section{Dimensiones homogeneas}
\begin{frame}{Dimensiones homogeneas}
\begin{exampleblock}{}
En ingenieria y ciencias, todas las ecuaciones deben ser dimensionalmente homogeneas, esto quiere decir que las cantidades que se suman o restan en la equacion tienen las mismas dimensiones. Por ejemplo, en la ecuaci\'on de Bernoulli para flujo incompresible:
$$
p + \frac{1}{2} \rho V^2 + \rho g Z = constante
$$
todos los terminos de la ecuaci\'on tienen dimensiones de presi\'on $[ML^{-1}T^{-2}]$.
\end{exampleblock}{}
\end{frame}


%\section{Introducci\'on}
%\begin{frame}{Que es la mec\'anica de fluidos}
%\begin{exampleblock}{}
%La mec\'anica de fluidos es la ciencia que hace parte de la mecánica clásica la cual estudia los fluidos estáticos o en movimiento y su interacción con otros objetos o fluidos.
%\end{exampleblock}
%\end{frame}
%
%\begin{frame}{Ramas de la mecánica de fluidos}
%\begin{block}{Hidrodinámica}
%Estudia fluidos en movimiento que pueden ser considerados incompresibles e.g. agua y gases a bajas velocidades.
%\end{block}
%\begin{figure}
%\centering
%\includegraphics[width=0.33\textwidth]{hydro1}
%\includegraphics[width=0.33\textwidth]{hydro2}
%\includegraphics[width=0.33\textwidth]{hydro3}
%\end{figure}
%\end{frame}
%
%\begin{frame}{Ramas de la mecánica de fluidos}
%\begin{block}{Hidráulica}
%Estudia el movimiento de líquidos en tuberías y canales abiertos (e.g. Rios).
%\end{block}
%\begin{figure}
%\centering
%\includegraphics[width=0.45\textwidth]{hydra1}
%\includegraphics[width=0.45\textwidth]{hydra2}
%\end{figure}
%\end{frame}
%
%\begin{frame}{Ramas de la mecánica de fluidos}
%\begin{block}{Dinámica de gases}
%Estudia el flujo de fluidos sometidos a cambios importantes de la densidad, e.g. flujo de gases a alta velocidad.
%\end{block}
%\begin{figure}
%\centering
%\includegraphics[width=0.45\textwidth]{gasd1}
%\includegraphics[width=0.45\textwidth]{gasd2}
%\end{figure}
%\end{frame}
%
%\begin{frame}{Ramas de la mecánica de fluidos}
%\begin{block}{Aerodinámica}
%Estudio del movimiento de gases, principalmente aire, alrededor de objetos e.g. cabina de un avión, cohetes y automóviles.
%\end{block}
%\begin{figure}
%\centering
%\includegraphics[width=0.45\textwidth]{aero1}
%\includegraphics[width=0.45\textwidth]{aero2}
%\end{figure}
%\end{frame}
%
%\begin{frame}{Estado de los fluidos}
%De la física, los fluidos existen en tres diferentes estados:
%\begin{itemize}
%\item Solido
%\item Liquido 
%\item Gas
%\item Plasma (fluido a altas temperaturas)
%\end{itemize}
% 
%\begin{exampleblock}{Liquido vs solido}
%La diferencia entre un liquido y un solido es en la habilidad de resistir esfuerzos que tienden a cambiar su forma. Por lo tanto, mientras un solido es capaz de resistir esfuerzos cuando se deforma, \textbf{un fluido se deforma continuamente y sin parar cuando se aplica un esfuerzo} sin importar su magnitud. 
%\end{exampleblock}
%\end{frame}
%
%\begin{frame}{¿Que es un fluido?}
%\href{https://www.youtube.com/watch?v=G53gvVh230U}{\beamergotobutton{Link}}
%%\includemedia[
%%  width=0.6\linewidth,height=0.45\linewidth,
%%  activate=pageopen,
%%  flashvars={
%%    modestbranding=1 % no YT logo in control bar
%%   &autohide=1       % controlbar autohide
%%   &showinfo=0       % no title and other info before start
%%  }
%%]{}{https://www.youtube.com/watch?v=G53gvVh230U}   % Flash file
%\end{frame}
%
%\begin{frame}{Deformaciones en un fluido}
%%Sin consideramos la goma solida de la Figura~\ref{f1} la cual esta fija a una base y a la cual se le aplica una fuerza $F$ paralela a la base, la goma se deforma y su angulo de deformacion ($\alpha$), denominado \texttt{resistencia al esfuerzo} o \texttt{desplazamiento angular}, es proporcional a $F$. La fuerza actuante contraria a $F$ debido a la friccion entre la plaza superior y la goma, es igual $F=\tau A$ donde $\tau$ es el esfuerzo cortante y $A$ es el area de contacto entre ambas superficies. Si en lugar de la goma tuvieramos un liquido, las capas mas cercanas a la placa superior se moverian continuamente, sin importar la magnitud de la fuerza, y la velocidad decreceria con la profundidad.
%\centering
%\includegraphics[width=6cm]{fig1}
%
%\begin{itemize}
%\item $F$: Fuerza paralela a la base
%\item $\alpha$: \textbf{resistencia al esfuerzo} o \textbf{desplazamiento angular}. $\alpha$ $\propto$ $F$
%\item $F_f =\tau A$. Fuerza de fricci\'on, donde $\tau$ es el esfuerzo cortante y $A$ es el área de contacto
%\end{itemize}
%\begin{exampleblock}{}
%Si en lugar de la goma tuviéramos un liquido, las capas mas cercanas a la placa superior se moverían continuamente, sin importar la magnitud de la fuerza, y la velocidad decrecería con la profundidad.
%\end{exampleblock}
%\end{frame}
%
%%De la estatica de cuerpos podemos definir:
%%\begin{exampleblock}
%%Estres se define como la fuerza por unidad de area y se calcula como fuerza $F$ divida por el area $A$ sobre la cual actua la fuerza. Al descomponer la fuerza actuante, la fuerza normal dividad por el area es el estres normal y la componente tangencial por unidad de area es el esfuerzo cortante. Por ejemplo en un fluido en reposo (esfuerzo cortante igual a cero), el esfuerzo normal es la presion.
%%\end{exampleblock} 
%
%
%\begin{frame}{Líquidos vs gases}
%\vspace{-0.5cm}
%\begin{exampleblock}{}
%Líquidos y gases (o vapores) se diferencian en que al tener un liquido en un contenedor, el volumen del liquido permanece constante formando una superficie libre porque la fuerza de cohesión entre las moléculas es alta y las moléculas están cerca. En contraste, en un gas las moléculas se mueven aleatoriamente y están mas alejadas y tienden a ocupar todo el volumen del contenedor debido a la débil fuerza cohesiva de sus moléculas. Fluidos como el asfalto o lodos se comportan como solidos y líquidos dependiente de la magnitud de los esfuerzos aplicados. 
%\end{exampleblock}
%\centering
%\includegraphics[width=5.5cm]{liqGas}
%\end{frame}
%
%
%\begin{frame}{Fluido como un continuo}
%\centering
%\includegraphics[width=5.5cm]{conti}
%\begin{exampleblock}{}
%\begin{itemize}
%\item El diámetro de las moléculas es pequeño comparado con el espaciamiento entre ellas
%\item Las moléculas se mueven libremente
%\item A nivel microscópico, la densidad cambia constantemente. Estos cambios son despreciables para volúmenes relativamente grandes 
%\item Sin embargo, la densidad $rho$ cambia suavemente con el espacio y con el tiempo en aplicaciones reales: a esto se le llama \emph{continuo}. Lo contrario seria un análisis molecular
%\item Calculo diferencial es utilizado para analizar estas substancias
%\end{itemize}
%\end{exampleblock}
%\end{frame}
%
%\section{Historia de la mec\'anica de fluidos}
%\begin{frame}{Historia de la mec\'anica de fluidos}
%\begin{exampleblock}{}
%\begin{itemize}
%\item Uno de los grandes problemas de la humanidad ha sido el suministro de agua para uso domestico e irrigación.
%\item Las sociedades prehistóricas que perduraron fueron también aquellas que invirtieron en la construcción de sistemas de distribución de agua.
%\end{itemize}
%\end{exampleblock}
%\end{frame}
%
%\begin{frame}{Prehistoria}
%\vspace{-0.5cm}
%\begin{columns}
%\column{0.5\textwidth}
%Los acueductos del Imperio Romano (312 B.C.)\\
%\centering
%\includegraphics[width=0.55\textwidth]{romans}
%\centering
%\includegraphics[width=0.55\textwidth]{romans2}
%\column{0.5\textwidth}
%45 km de tubería en arcilla que transportaban agua a presión $>$1.5 Mpa (180 m cabeza de agua) ciudad Helenica de Pergamo, Turkey (283-133 B.C.) \\
%\centering
%\includegraphics[width=0.5\textwidth]{pergamo}
%\end{columns}
%\begin{block}{Arquímedes}
%El matemático Griego Arquímedes (285-212 B.C.) formulo y aplico el principio de flotación para saber la cantidad de oro en la corona del Rey Hieron de Siracusa.
%\end{block}
%\end{frame}
%
%\begin{frame}{Edad media}
%\begin{columns}
%\column{0.5\textwidth}
%Bombas de pistón fueron construidas para extraer el agua de las minas\\
%\centering
%\includegraphics[width=0.6\textwidth]{minas}
%\column{0.5\textwidth}
%Molinos de agua y de viento fueron desarrollados para moler granos y trabajar el hierro reemplazando la fuerza humana. Dio luego origen a la Revolución Industrial.\\
%\centering
%\includegraphics[width=0.9\textwidth]{wind}
%\end{columns}
%\end{frame}
%
%\begin{frame}{Renacimiento}
%\begin{columns}
%\column{0.8\textwidth}
%\vspace{-0.5cm}
%\begin{exampleblock}{}
%\begin{itemize}
%\item Importantes avances en la mecánica de fluidos gracias al desarrollo del Método Científico
%\item Simon Stevin (1548–1617), \emph{Galileo Galilei (1564–1642)}, Edme Mariotte (1620–1684), y Evangelista Torricelli (1608–1647) fueron los primero en aplicar el método al estudio de los fluidos para entender la distribución de presiones hidroestática. El matemático y filosofo \emph{Blaise Pascal (1623–1662)} mejoro e integro estos trabajos sobre hidroestatica.
%\item Benedetto Castelli (1577–1644) fue el primero en publicar el principio de continuidad en fluidos.
%\item \emph{Sir Isaac Newton (1643–1727)} aplic\'o las leyes de la mecánica a fluidos para explorar la inercia, la resistencia y la  viscosidad en fluidos.
%\end{itemize}
%\end{exampleblock}
%\column{0.2\textwidth}
%\vspace{-0.4cm}
%\begin{center}
%\includegraphics[width=0.7\textwidth]{gali}\\
%\includegraphics[width=0.7\textwidth]{pasca}\\
%\includegraphics[width=0.7\textwidth]{new}
%\end{center}
%\end{columns}
%\end{frame}
%
%\begin{frame}{Renacimiento}
%\begin{columns}
%\column{0.8\textwidth}
%\vspace{-0.4cm}
%\begin{exampleblock}{}
%\begin{itemize}
%\item \emph{Daniel Bernoulli (1700–1782)} y \emph{Leonard Euler (1707–1783)}, basados en los desarrollos de Newton, definieron las ecuaciones de conservación de la energía y de momentum.
%\item \emph{Hydrodynamica}, escrito por Bernoulli en 1738, es considerado el primer libro de mecánica de fluidos.
%\item Jean d’Alembert (1717–1789) desarrollo una expresión diferencial de la continuidad basado en la idea de las componentes de la velocidad y la aceleración.
%\item Debido a la dificultad de cuantificar propiedades de los fluidos, poco impacto tuvieron estos desarrollos en la ingeniería. 
%\end{itemize}
%\end{exampleblock}
%\column{0.2\textwidth}
%\vspace{-0.8cm}
%\begin{center}
%\includegraphics[width=0.8\textwidth]{berno}\\
%\includegraphics[width=0.8\textwidth]{hydrody}\\
%\includegraphics[width=0.8\textwidth]{euler}
%\end{center}
%\end{columns}
%\end{frame}
%
%\begin{frame}{Siglo XIX (Europa)}
%\vspace{-0.4cm}
%\begin{exampleblock}{}
%\begin{itemize}
%\item Se introduce C\'alculo en el pensum de las escuelas de ingeniería y esto genera grandes desarrollos durante este siglo:
%\item \emph{Jean Poiseuille (1799–1869)} midió flujo en flujos capilares para diferentes fluidos
%\item En Alemania, \emph{Gotthilf Hagen (1797–1884)} hizo experimentos para diferenciar flujo laminar y flujo turbulento en tuberías.
%\end{itemize}
%\end{exampleblock}
%\begin{center}
%\includegraphics[width=0.26\textwidth]{poise}\hspace{2cm}
%\includegraphics[width=0.26\textwidth]{hagen}
%\end{center}
%\end{frame}
%
%\begin{frame}{Siglo XIX (Europa)}
%\vspace{-0.4cm}
%\begin{exampleblock}{}
%\begin{itemize}
%\item En Inglaterra, \emph{Lord Osborne Reynolds (1842–1912)} continuo el trabajo de Hagen y desarrollo el numero adimensional que lleva su nombre.
%\item En paralelo, \emph{Louis Navier (1785–1836)} y \emph{George Stokes (1819–1903)} establecieron las ecuaciones del movimiento de los fluidos. Este ultimo incluyo la fricción.
%\item \emph{William Froude (1810–1879)} demostró la importancia de la modelaci\'on física.
%\end{itemize}
%\end{exampleblock}
%\begin{center}
%\includegraphics[width=0.2\textwidth]{reyn}
%\includegraphics[width=0.2\textwidth]{navier}
%\includegraphics[width=0.2\textwidth]{stoke}
%\includegraphics[width=0.2\textwidth]{froude}
%\end{center}
%\end{frame}
%
%
%\begin{frame}{Finales del XIX (Estados Unidos)}
%\vspace{-0.4cm}
%\begin{exampleblock}{}
%\begin{itemize}
%\item \emph{James Francis (1815–1892)} y \emph{Lester Pelton (1829–1908)}, aplicando la teoría hasta ahora desarrollada, construyeron y comercializaron turbinas.
%\item \emph{Clemens Herschel (1842–1930)} invento el \emph{tubo Venturi} para medir caudales de flujo.
%\end{itemize}
%\end{exampleblock}
%\begin{center}
%\includegraphics[width=0.3\textwidth]{franc}
%\includegraphics[width=0.3\textwidth]{pelton}
%\includegraphics[width=0.3\textwidth]{ventu}
%\end{center}
%\end{frame}
%
%\begin{frame}{Finales del XIX (Inglaterra)}
%\vspace{-0.4cm}
%\begin{exampleblock}{}
%William Thomson, Lord Kelvin (1824–1907), William Strutt, Lord Rayleigh (1842–1919), and Sir Horace Lamb (1849–1934) fueron pioneros en investigar: análisis dimensional, flujo irrotacional, vortices, cavitaci\'on y olas. Exploraron las relaciones entre la mecánica de fluidos, la termodinámica y la transferencia de calor.
%\end{exampleblock}
%\end{frame}
%
%\begin{frame}{Primera mitad del siglo XX}
%\begin{columns}
%\column{0.3\textwidth}
%Los autodidactas \emph{hermanos Wright (1903)} inventaron el avión usando conceptos de la mecánica de fluidos y haciendo experimentos.\\
%\centering
%\includegraphics[width=1.1\textwidth]{wrig}
%\column{0.7\textwidth}
%\vspace{-1cm}
%\begin{itemize}
%\item El aleman \emph{Ludwig Prandtl (1875–1953)}, demostró que los fluidos pueden dividirse en dos partes: una capa delgada cerca a la pared en donde la fricción es importante llamada capa limite, y otro capa en donde la fricción es despreciable y las ecuaciones simplificadas de Euler y Bernoulli pueden aplicarse.
%\item Con base en las teorías de Prandtl, Theodor von Kármán (1881–1963), Paul Blasius (1883–1970), Johann Nikuradse (1894–1979) y otros avanzaron en aplicaciones de la hidráulica y la aerodinámica.
%\end{itemize}
%\end{columns}
%\end{frame}
%
%\begin{frame}{Segunda mitad del siglo XX}
%\vspace{-0.4cm}
%\begin{exampleblock}{}
%\begin{itemize}
%\item Los a\~nos dorados de las aplicaciones de la mecánica de fluidos: grandes desarrollos en sectores de la aeronáutica, la industria química y los recursos hidráulicos.
%\item Importantes investigaciones avanzaron con el invento del computados digital. Gracias a esto, fue posible trabajar en problemas complejos como la circulación global de la atmósfera y de los océanos, y la optimización en el diseño de turbinas.
%\end{itemize}
%\end{exampleblock}
%\begin{center}
%\includegraphics[width=0.4\textwidth]{compu}
%\includegraphics[width=0.4\textwidth]{hoov}
%\end{center}
%\end{frame}
%
%\begin{frame}{En la actualidad}
%\vspace{-0.4cm}
%\begin{exampleblock}{}
%\begin{itemize}
%\item Modelos de predicción hidrológica y calidad de agua
%\item Modelos de circulación global y oceánica a alta resolución
%\item Optimización en el diseño de estructuras y sistemas de regadío y distribución
%\end{itemize}
%\end{exampleblock}
%\begin{center}
%\includegraphics[width=0.4\textwidth]{rivba}
%\includegraphics[width=0.4\textwidth]{oceanc}
%\includegraphics[width=0.4\textwidth]{irriga}
%\end{center}
%\end{frame}
%
%
%
%\section{Sistemas de unidades}
%
%\begin{frame}{Unidades primarias}
%\centering
%\includegraphics[width=12cm]{primu}
%\end{frame}
%
%\begin{frame}{Unidades secundarias}
%\centering
%\includegraphics[width=12cm]{secdim}
%\end{frame}




%\begin{frame}{Informaci\'on}
%\begin{exampleblock}{}
%\begin{itemize}
%\item \alert{Oficina}: Edificio Laboratorio de Hidr\'aulica, oficina 304 
%\item \alert{Email}: lmoralesm@unal.edu.co
%\end{itemize}
%\end{exampleblock}
%\end{frame}
%
%\begin{frame}{La asignatura}
%\begin{itemize}
%\item Identificaci\'on de la asignatura: \alert{2015966}
%\item Duraci\'on clase presencial: \alert{5 hrs/semana}
%\item No. de semanas: \alert{16}
%\item No. de cr\'editos: \alert{4}
%\item ¿Esta asignatura es validable?: \alert{No}
%\item ¿Asignatura de libre elecci\'on?: \alert{No}
%\item Planes de estudio a los que se asocia la asignatura
%\begin{itemize}
%\item \alert{2540 Ingenier\'ia Agr\'icola Componente C}
%\item \alert{2541 Ingenier\'ia Civil Componente C}
%\end{itemize}
%\item Prerequisitos
%\begin{itemize}
%\item \alert{C\'alculo en varias variables}
%\item \alert{Ecuaciones diferenciales}
%\item \alert{Est\'atica}
%\end{itemize}
%\end{itemize}
%\end{frame}
%
%\begin{frame}{P\'agina de la asignatura}
%\begin{block}{P\'agina de la asignatura}
%\href{https://lamhydro.github.io/fluidMechanics/}{https://lamhydro.github.io/fluidMechanics/}
%\end{block}
%\end{frame}
%
%\section{Horario de clases}
%\begin{frame}{Horario de clases}
%\centering
%\includegraphics[width=9cm]{horarioClases}
%\end{frame}
%
%\section{Descripci\'on}
%\begin{frame}{Descripci\'on}
%\vspace{-0.8cm} 
%\begin{columns}
%\column{0.6\textwidth}
%\begin{exampleblock}{}
%\small
%La asignatura mecánica de fluidos se centra en el estudio y análisis de las propiedades físicas más relevantes de los fluidos a partir de los aspectos fundamentales de la física, el cálculo y las ecuaciones diferenciales. Con ello el curso se centra en el estudio y aplicación de \alert{cuatro principios: de Pascal, de conservación de la masa, de conservación de la energía, momentum lineal y angular}, principios que se aplican a diferentes condiciones de contorno a los cuales están sometidos los fluidos. Finalmente se hace una introducción al análisis dimensional y semejanza hidráulica.
%\end{exampleblock}
%\column{0.4\textwidth}
%\begin{center}
%\includegraphics[width=\textwidth]{fm}
%\includegraphics[width=\textwidth]{fm1}
%\end{center}
%\end{columns}
%\end{frame}
%
%
%\section{Conceptos previos necesarios}
%\begin{frame}{Conceptos previos necesarios}
%\begin{exampleblock}{}
%El curso requiere el uso de herramientas matemáticas para la solución de \alert{ecuaciones diferenciales} y el entendimiento básico de los teoremas fundamentales del \alert{cálculo multivariado}, así como los conocimientos básicos de Estática. Es deseable que los estudiantes tengan conocimientos básicos en programación de computadores y el manejo de herramientas de métodos numéricos.
%\end{exampleblock}
%\end{frame}
%
%\section{Objetivo}
%\begin{frame}{Objetivo}
%\begin{columns}
%\column{0.6\textwidth}
%\begin{exampleblock}{}
%Dar a conocer las \alert{leyes físicas fundamentales} que tienen que ver con el comportamiento de los fluidos y sus aplicaciones a \alert{problemas típicos de la Ingeniería}. Al finalizar el curso, el estudiante debe estar en capacidad de aplicar las ecuaciones básicas de mecánica de fluidos.
%\end{exampleblock}
%\column{0.4\textwidth}
%\begin{center}
%\includegraphics[width=\textwidth]{fmo1}
%\end{center}
%\end{columns}
%\end{frame}
%
%\section{Contenido general de la materia}
%
%\begin{frame}{1. Propiedades de los fluidos}
%\begin{exampleblock}{}
%\begin{itemize}
%\item Sistemas de unidades
%\item Transformación de unidades
%\item Definición de presión y esfuerzo de corte
%\item Ley de viscosidad de Newton
%\item Tipos de fluidos y tipos de flujo
%\item Propiedades de los fluidos
%\item Ecuación de estado de los gases y gases perfectos
%\end{itemize}
%\end{exampleblock}
%\end{frame}
%
%\begin{frame}{2. Est\'atica de los fluidos}
%\vspace{-0.9cm}
%\begin{columns}
%\column{0.6\textwidth}
%\begin{exampleblock}{}
%\begin{itemize}
%\small
%\item Fluidos en reposo, escala de medida de la presión y de temperatura
%\item Ecuación fundamental de la estática de los fluidos, fluido incompresible, fluido compresible, principio de Pascal
%\item Aparatos medidores de presión
%\item Fuerzas sobre cuerpos sumergidos: superficies planas, superficies curvas, principios de flotación
%\item Equilibrio relativo de fluidos en movimiento
%\item Aceleración lineal uniforme, rotación uniforme alrededor de un eje vertical
%\item \alert{Laboratorios: Aparatos medidores de presión, Ensayos de flotación, Propiedades de los fluidos (banco de estática), Aparato de Reynolds}
%\end{itemize}
%\end{exampleblock}
%\column{0.4\textwidth}
%\begin{center}
%\includegraphics[width=\textwidth]{fme}
%\includegraphics[width=\textwidth]{fme1}
%\end{center}
%\end{columns}
%\end{frame}
%
%\begin{frame}{3. Cinem\'atica de los fluidos}
%\vspace{-0.9cm}
%\begin{columns}
%\column{0.6\textwidth}
%\begin{exampleblock}{}
%\begin{itemize}
%\item Generalidades, propiedades cinemáticas del flujo
%\item Métodos para describir el movimiento de un fluido: método de Lagrange y método de Euler
%\item Flujo volumétrico y flujo másico, línea de corriente y clasificación de los flujos
%\item Teorema de Transporte de Reynolds, ecuación de continuidad para un volumen de control, continuidad en un punto
%\item Flujo potencial y función de corriente, relación entre flujo potencial y función de corriente
%\item \alert{Laboratorio: Hele-Shaw}
%\end{itemize}
%\end{exampleblock}
%\column{0.4\textwidth}
%\begin{center}
%\includegraphics[width=\textwidth]{fmk}
%\end{center}
%\end{columns}
%\end{frame}
%
%\begin{frame}{4. Din\'amica de los fluidos}
%\vspace{-1.2cm}
%\begin{columns}
%\column{0.6\textwidth}
%\begin{exampleblock}{}
%\begin{itemize}
%\footnotesize
%\item Ecuación de energía para flujos incompresibles y significado físico
%\item Concepto de línea de energía, línea de gradiente hidráulico y potencia hidráulica
%\item Aplicaciones de la ecuación de energía a sistemas de conducción: tuberías, canales, sistemas de bombeo, sistemas de generación hidroeléctrica
%\item Medidores de caudal: orificios, tubo Pitot, tubo Venturi, medidor de codo
%\item Cantidad de movimiento: cantidad de movimiento lineal aplicado a: estructuras hidráulicas como compuertas, vertederos, salto hidráulico, accesorios en tuberías, álabes fijos y móviles. 
%\item Cantidad de movimiento angular.
%\scriptsize
%\item \alert{Laboratorios: Línea de energía y línea de gradiente hidráulico, Flujo compresible, Tubo Pitot, Aparatos medidores de caudal: orificio, tubo Venturi.}
%\end{itemize}
%\end{exampleblock}
%\column{0.4\textwidth}
%\begin{center}
%\includegraphics[width=\textwidth]{fmd}
%\includegraphics[width=\textwidth]{fmd1}
%\end{center}
%\end{columns}
%
%\end{frame}
%
%\begin{frame}{5. Analisis dimensional}
%\begin{columns}
%\column{0.5\textwidth}
%\begin{exampleblock}{}
%\begin{itemize}
%\item Ecuación dimensionalmente homogénea
%\item Parámetros adimensionales relevantes en mecánica de fluidos
%\item Obtención de ecuaciones y teorema pi de Buckingham
%\item Determinación de los grupos adimensionales y leyes de semejanza
%\item Clasificación de los modelos físicos y aplicaciones
%\end{itemize}
%\end{exampleblock}
%\column{0.5\textwidth}
%\begin{center}
%\includegraphics[width=\textwidth]{dimana}
%\end{center}
%\end{columns}
%\end{frame}
%
%\section{Evaluaci\'on del curso}
%\begin{frame}{Evaluaci\'on del curso}
%\centering
%\includegraphics[width=9cm]{eval}
%\end{frame}
%
%\section{Referencias}
%\begin{frame}{Referencias}
%\begin{block}{Recomendadas}
%\begin{enumerate}
%\item \textbf{Çengel, Y. A., \& Cimbala, J. M. (2010). Fluid Mechanics: Fundamentals and Applications, McGraw-Hill Higher Education.}
%\item \textbf{Duarte, C. A. (2017). Mecánica de fluidos e hidráulica, Universidad Nacional de Colombia, Facultad de Ingeniería,Departamento de Ingeniería Civil y Agrícola.}
%\item \textbf{White, F. (2015). Fluid Mechanics, McGraw-Hill Higher Education.}
%\end{enumerate}
%\end{block}
%\end{frame}
%
%\begin{frame}{Referencias}
%\begin{block}{Otros}
%\begin{enumerate}
%\small
%\item Franzini, J. B., \& Finnemore, E. J. (1997). Fluid Mechanics with Engineering Applications: McGraw-Hill.
%\item Liu, C. (2013). Schaum’s Outline of Fluid Mechanics and Hydraulics, 4th Edition: McGraw-Hill Education.
%\item Mott, R. L. (2006). Applied Fluid Mechanics: Pearson Prentice Hall.
%\item Potter, M. C., Wiggert, D. C., \& Ramadan, B. H. (2016). Mechanics of Fluids, SI Edition: Cengage Learning.
%\item Pritchard, P. J. (2010). Fox and McDonald's Introduction to Fluid Mechanics, 8th Edition: John Wiley \& Sons.
%\item Roberson, J. A., \& Crowe, C. T. (1999). Engineering Fluid Mechanics: Jaico Publishing House.
%\item Shames, I. H. (1992). Mechanics of Fluids: McGraw-Hill.
%\item Sotelo, G. (1974). Hidráulica general: fundamentos: Limusa.
%\item Street, R. L., Watters, G. Z., \& Vennard, J. K. (1995). Elementary Fluid Mechanics: Wiley.
%\item Streeter, V. L., \& Wylie, E. B. (1979). Fluid mechanics: McGraw-Hill.
%\end{enumerate}
%\end{block}
%\end{frame}

%\begin{frame} [allowframebreaks]\frametitle{References}
               
%        \bibliographystyle{apalike}
%        \bibliography{bibfile}
%\end{frame}

\end{document}


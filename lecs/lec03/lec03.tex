\documentclass[10pt, oneside]{article} 
\usepackage{amsmath, amsthm, amssymb, calrsfs, wasysym, verbatim, bbm, color, graphics, graphicx, geometry}
\usepackage[most]{tcolorbox}
\usepackage{xcolor}
\usepackage{framed}
\colorlet{shadecolor}{blue!15}
\graphicspath{ {./} }

\geometry{tmargin=.75in, bmargin=.75in, lmargin=.75in, rmargin = .75in}  

\newcommand{\R}{\mathbb{R}}
\newcommand{\C}{\mathbb{C}}
\newcommand{\Z}{\mathbb{Z}}
\newcommand{\N}{\mathbb{N}}
\newcommand{\Q}{\mathbb{Q}}
\newcommand{\Cdot}{\boldsymbol{\cdot}}

\newtheorem{thm}{Theorem}
\newtheorem{defn}{Definition}
\newtheorem{conv}{Convention}
\newtheorem{rem}{Remark}
\newtheorem{lem}{Lemma}
\newtheorem{cor}{Corollary}
\newtheorem{exa}{Example}


\title{Clase \# 3: Presion y estatica de los fluidos} 
\author{\textbf{Luis Alejandro Morales}\\ \vspace{0.4cm} Profesor Asistente \\ Universidad Nacional de Colombia-Bogot\'a\\Facultad de Ingenier\'ia \\ Departamento de Ingenieria Civil y Agr\'icola}
\date{Periodo 2022-II}

\begin{document}

\maketitle
\tableofcontents

\vspace{.25in}


\section{Presi\'on}
En mecanica de fluidos, la \textbf{presi\'on} es definida como la fuerza normal ejercida por un fluido por unidad de area. En otras areas, la presion ejercida sobre un solido se conoce como el \textbf{esfuerzo normal}. En SI, las unidades de la presi\'on  son el \textbf{pascal}(Pa) donde $1\ Pa = 1\ N/m^2$. Otras unidades de presion usadas comunmente son:
\begin{flalign*}
1\ bar\ &=\ 10^5\ Pa\ =\ 0.1\ MPa\ =\ 100\ kPa \\
1\ atm\ &=\ 101325\ Pa\ =\ 101.325\ kPa\ =\ 1.01325\ bars \\
1\ kgf/cm^2 \ &=\ 9.807\ N/cm^2\ =\ 9.807E4\ N/m^2\ =\ 9.807E4\ Pa \\
&=\ 0.9807\ bar \\
&=\ 0.9679\ atm 
\end{flalign*}

En sistema ingles, las unidade de la presi\'on son libra fuerza por puldaga cuadrada ($lbf/in^2$ o $psi$). Algunas equivalencias entre los dos sistemas son:
\begin{flalign*}
1\ atm\ &=\ 14.696\ psi \\
1\ kgf/cm^2 \ &=\ 14.223\ psi 
\end{flalign*}

Algunas definiciones son:
\begin{itemize}
\item \textbf{presion absoluta} $P_{abs}$: Es la presion actual sobre un cuerpo y es medida con respecto al vacio absoluto (presion cero). La mayoria de los medidores de presion son calibrados para medir la presion tomando como cero la presi\'on atmosferica local.
\item \textbf{presion de manometro} $P_{gage}$: Es la diferencia entre la presion absoluta y la presion atmosferica local. 
\item \textbf{presion de vacio} $P_{vac}$: Se presenta cuando una presion esta por debajo de la presion atmosferica ($P_{gage}<0$). 
\end{itemize}

La relacion entre estas presiones es (ver figura~\ref{pres1}):
\begin{equation}
P_{gage} = P_{abs}-P_{atm}
\end{equation}

\begin{equation}
P_{vac} = P_{atm}-P_{abs}
\end{equation}

% Fig 3.3 Cengel
\begin{figure}[h]
\centering
\includegraphics[width=10cm]{pres1}
\caption{Presi\'on atmosferica, de manometro y de vacio}
\label{pres1}
\end{figure}

Es comun encontrar por ejemplo que la presion de manometro tipica de una llanta de un carro es 32.0 $psi$, la cual es tomada con respecto a la presion atmosferica. Si la presion atmosferica en el sitio en donde se encuentra el carro es de 14.3 $psi$, la presion absoluta seria $32.0\ +\ 14.3\ =\ 46.3\ psi$. Note que en algunos problemas las unidades $psig$ hacen referencia a presion de manometro mientras que las unidades $psia$ hacen referencia a presion absoluta. 

\subsection{Presi\'on en un punto}
Por definicion, la presion en un punto en un fluido es la misma en todas las direcciones, por lo tanto la presion no puede considerarse como una cantidad vectorial y es entonces una cantidad scalar. Esto se puede demostrar si se hace un analisis de fuerzas sobre el elemento en equilibrio de la figura~\ref{ppoint} en donde la profundidad del elemento es $\Delta y = 1$. Aplicando la segunda ley de Newton y haciendo un analisis de fuerzas sobre las superficies del elemento:
% Fig 3.5 Cengel
\begin{figure}[h]
\centering
\includegraphics[width=10cm]{ppoint}
\caption{Fuerzas actuantes sobre un elemento de fluido en equilibrio.}
\label{ppoint}
\end{figure}


\begin{equation}
\begin{split}
\sum F_x &= ma_x = 0: \quad P_1 \Delta y \Delta z - P_3 \Delta y l \sin \theta = 0 \\
\sum F_z &= ma_z = 0: \quad P_2 \Delta y \Delta x - P_3 \Delta y l \cos \theta - \frac{1}{2}\rho g \Delta x \Delta y \Delta z = 0
\label{ppo}
\end{split}
\end{equation}

donde $\rho$ es la densidad y $W = mg = \frac{1}{2} \rho g \Delta x \Delta y \Delta z$ es el peso del elemento. Teniendo en cuenta que $\Delta x = l \cos \theta$ y $\Delta z = l \sin \theta$, reemplazando en la ecuacion~\ref{ppo} y simplificando:

\begin{equation}
\begin{split}
P_1 - P_3 = 0 \\
P_2 - P_3 - \frac{1}{2}\rho g \Delta z = 0
\label{ppo1}
\end{split}
\end{equation}
El ultimo termino de la ecuaci\'on~\ref{ppo1} se elimina teniendo en cuenta que cuando el elemento se vulve infinitesimal y se reduce a un punto $\Delta z \rightarrow 0$. Por tanto:

\begin{equation}
\begin{split}
P_1 = P_2 = P_3 = P
\label{ppo2}
\end{split}
\end{equation}

Con esto concluimos que la presion $P$ en un punto en un fluido tiene la misma magnitud en todas las direcciones. Esto se aplica para un fluido en moviento o en reposo teniendo en cuenta que la presion es un escalar. 

\subsection{Variaci\'on de la presi\'on con la profundidad}
Es conocido que la presion de un fluido en reposo no cambia en direccion horizontal y su cambio es en direccion vertical. Por esto, la presion en un fluido incrementa con la profundidad ya que este incremento implica mayor cantidad de fluido y por tanto mayor peso lo cual es balanceado con un incremento de la presi\'on.

Para obtener una relacion de la variaci\'on de la presion con la profundidad, analicemos las fuerzas actuantes sobre el elemento en equilibrio de la figura~\ref{pres2} cuya profundidad es $\Delta y =1$. 

% Fig 3.7 Cengel
\begin{figure}[h]
\centering
\includegraphics[width=10cm]{pres2}
\caption{Diagrama de cuerpo libre de un elemento rectangular de fluido en equilibrio.}
\label{pres2}
\end{figure}

Asumiendo que la densidad $\rho$ del fluido es constante, el balance de fuerzas en la direccion $z$ es:
$$
\sum F_z = m a_z = 0: \quad P_1 \Delta x \Delta y - P_2 \Delta x \Delta y - \rho g \Delta x \Delta y \Delta z = 0
$$
donde $W = mg = \frac{1}{2} \rho g \Delta x \Delta y \Delta z$ es el peso del elemento y $\Delta z = z_2 - z_1 $. Dividiendo por $\Delta x \Delta y$, tenemos:

\begin{equation}
\Delta P = P_2 - P_1 = -\rho g \Delta z = -\gamma_s \Delta z
\label{ppr1}
\end{equation}

donde $\gamma_s = \rho g$ es el peso especifico  del fluido. Otra manera de expresar la ecuacion~\ref{ppr1}anterior es:

\begin{equation}
\Delta P_{below} = P_{above} + \gamma_s |\Delta z|
\label{ppr2}
\end{equation}

donde "below" indica el punto mas bajo mientras que "above" indica el punto mas alto. Debido a que la densidad de los gases $\approx\ 0$, por ejemplo, la presion en una habitacion es uniforme ya que el peso del gas es muy bajo por lo que la ecuacion~\ref{ppr1} se convierte en $\Delta P=0$. 
Si tomamos el punto "above" sobre la superficie de el liquido a superficie abierta, cuya presion es la atmosferica $P_{atm}$, la presion a una profundidad $h$ (medida desde la superficie), la presion es:

\begin{equation}
P = P_{atm} + \rho g h \quad \text{or} P_{gage}=\rho g h
\label{ppr3}
\end{equation}

Como los fluidos son esencialmente inconpresibles, la variaci\'on de $\rho$ es despreciable con respecto a la profundidad. Cuando se requiere una alta precision en el calculo de $P$  debido a cambios fuertes de temperature en fluidos, es necesario saber como $\rho$ cambia con la temperatura. Ademas, cuando se requiere calcular  $P$ a grandes profundidades en el oceano, es importante determinar como cambia a la densidad con la profundidad.

Para fluidos cuya densidad cambia significantemente con la elevacion, una relacion de la variacion de la presion con respecto a la elevacion es obtenida dividiendo la ecuacion~\ref{ppr1} por $\Delta z \rightarrow 0$, es:

\begin{equation}
\frac{dP}{dz} = -\rho g
\label{ppr4}
\end{equation}

Noten que $dP$ es negativa cuando $dz$ es positiva teniendo en cuenta que la presion decrease hacia arriba.Si $\rho$ es conocida con la elevacion, la diferencia de presion entre dos puntos 1 y 2 (ver figura~\ref{pres2} se determina como:

\begin{equation}
\Delta P = P_2 - P_1 = -\int_1^2 \rho g dz
\label{ppr5}
\end{equation}

Como lo habiamos mencionado anteriormente, en un fluido en reposo la presion sobre cualquier tipo de superficie cambia unicamente con la profundidad. En la figura~\ref{pres3} vemos que la presion en los puntos A, B, C, D, E, F y G sobre superficies de diferentes formas es la misma, ya que estan conectados por el mismo liquido y estan a la misma profundidad $h$. Es ademas importante recordar que la presion es siempre normal a la superficie. Por otro lado la presion sobre los puntos H e I no es la misma porque estan a diferente profundidad y ademas no estan conectados por el mismo fluido. 

% Fig 3.10 Cengel
\begin{figure}[h]
\centering
\includegraphics[width=10cm]{pres3}
\caption{Presion sobre puntos sobre superficies de diferentes formas y a diferentes profundidades.}
\label{pres3}
\end{figure}

Como consequencia de que la presion es constante en direccion horizontal, tenemos que \textit{la presion aplicada sobre un fluido confinado en un contenedor, es transmitida igualmente a todas las partes del contenedor y actua perpendicular a las paredes del mismo}. Esto es conocido como la \textbf{Ley de Pascal}. Dicho de otra manera \textit{un cambio en la presion en cualquier punto de un fluido en reposo es transmitido igualmente a todos los puntos del fluido}. La ley de pascal tiene muchas aplicaciones como por ejemplo el sistema de frenos en vehiculos, los elevadores hidraulicos para levantar cargas pesadas, entre otros. Si analisamos la figura~\ref{pres4}, tenemos que $P_1 \ =\ P_2$ ya que estan a un mismo nivel, esto con lleva a la siguiente relacion de fuerzas:

% Fig 3.11 Cengel
\begin{figure}[h]
\centering
\includegraphics[width=10cm]{pres4}
\caption{Ley de Pascal en un elevador hidraulico.}
\label{pres4}
\end{figure}

\begin{equation}
P_1 = P_2 \quad \rightarrow \quad \frac{F_1}{A_1}=\frac{F_2}{A_2} \quad \rightarrow \quad \frac{F_2}{F_1}=\frac{A_2}{A_1}
\label{ppr6}
\end{equation}

donde la relacion $A_2 / A_1$ es conocida como la \textit{ventaja mecanica} de un elevador hidraulico.

\section{Medidores de presi\'on}
\subsection{Barometro}
La presi\'on atmospherica es medida por un aparato llamado \textbf{barometro} por lo que la presion atmosferica es usualmente conocida como la presion baramometrica. El barometro fue inventado por Evangelista Torricelli (1608-1647) y consiste en invertir un tubo de ensayo lleno de mercurio dentro de un recipiente con mercurio abierto a la atmosfera (ver figura~\ref{baro1}). La presion en el punto B es igual a la presion atmoferica, mientras la presion en el punto C puede considerarse igual zero. Escribiendo el balance de fuerzas sobre la columna de mercurio:

% Fig 3.12 Cengel
\begin{figure}[h]
\centering
\includegraphics[width=10cm]{baro1}
\caption{El barometro}
\label{baro1}
\end{figure}


\begin{equation}
P = \rho g h
\label{bb1}
\end{equation}
donde $\rho$ es la densidad del mercurio. Note que la altura $h$ de la columns es siempre la misma independiente del diametro del tubo. 

Algunas definiciones:
\begin{itemize}
\item \textbf{atmosfera estandard}: Presion producida por una columna de mercurio de 760 $mm$ de $\rho_{Hg} = 13595\ kg/m^3$ a 0$^oC$ bajo estandard $g=9.807\ m/s^2$. El equivalente en columna de agua seria de 10.3 $m$.
\item \textbf{Torr}: En honor a Torricelli, la unidad $mmHg$ es conocida como $torr$. Por esto, $1\ atm = 760\ torr$ y $1\ torr = 133.3\ Pa$. 
\end{itemize}

Algunas ideas importantes:
\begin{itemize}
\item La $P_{atm}$ disminuye con la altura. Por eso mientras que $P_{atm}=101.325 kPa$, la presion a altitudes como 1000, 2000, 5000 y 10000 y 20000 metros es 89.88, 79.50, 54.05, 26.5 and 5.53 $kPa$, respectivamente.
\item Si $P_{atm}$ depende del peso del aire arriba de una posicion determinada, esta no solo cambia con la altitud, tambien con las condiciones climaticas.
\item Como la temperatura y la presion disminuyen con la altura, cocinar hervir agua en sitios en altas altitudes toma mayor tiempo.
\item Es comun el sangrado nasal en altas altitudes porque la diferencia entre la presion sanguinea y la presion atmosferica se hace mayor por lo que los vasos sanquineos de la nariz son incapaces de soportar este esfuerzo adicional y terminan rompiendose. 
\item Como en altas altitudes la densidad del aire es mas baja, la cantidad de oxigeno por unidad de volumen es menor, por eso nos cansamos mas rapidamente en estos lugares y experimenmos dificultad al respirar.
\end{itemize}

\subsection{El manometro}
De acuerdo con la ecuaci\'on~\ref{ppr1}, el cambio de elevacion $-\Delta z$ en un fluido en reposo es igual a $\Delta P/\rho g$, lo cual sugiere que la columna de un fluido puede ser usada para calcular las diferencias de presion. El \textbf{manometro} es un aparato que esta basado en este principio y por lo tanto es usado para medir differencias de presion. Un manometro es un tubo de plastico o vidrio en forma de U el cual contiene usualmente agua, mercurio, alcohol o aceite (ver figura~\ref{mano1}). Cuando las diferencias de presion son muy  altas, se prefiere un fluido pesado como el mercurio.

\begin{figure}[h]
\centering
\includegraphics[width=10cm]{mano1}
\caption{Manometro en forma de U}
\label{mano1}
\end{figure}

Consideremos el manometro conectado al tanque con gas de la figura~\ref{mano2}. Teniendo en cuenta que los efectos gravitaciones sobre los gases son despreciables, la presion en cualquier punto del tanque es la misma incluyendo la presion en 1 $P_1$. Se sabe ademas que la presion no varia en direccion horizontal en un fluido, por lo tando, $P_2\ =\ P_1$. Como la altura $h$ de fluido esta en equilibrio estatico y esta abierta a la atmosfera:

% Fig 3.19 Cengel
\begin{figure}[h]
\centering
\includegraphics[width=10cm]{mano2}
\caption{Manometro b\'asico.}
\label{mano2}
\end{figure}

\begin{equation}
P_2 = P_{atm} + \rho g h
\label{ma1}
\end{equation}

donde $\rho$ es la densidad del fluido del manometro. A pesar que el area transversal del manometro no cambia $h$, el diametos del tubo debe ser lo suficientemente grande para reducir el efecto capilar.

Algunos problemas en ingenieria involucran manometros con multiples fluidos de diferentes densidades ubicados uno sobre otro. Recuerde que para resolver cualquier problema de manometros:
\begin{enumerate}
\item El cambio de presion en una columna de fluido $h$ es: $\Delta P = \rho gh$.
\item La presion en un fluido incrementa hacia abajo y disminuye hacia arriba ($P_{down} > P_{top}$).
\item Dos puntos conectados por un fluido continuo en reposo sobre el mismo plano horizontal tienen la misma presion. 
\end{enumerate}

Cuando existen differentes tipos de fluidos conectados continuamente y en reposo se puede calcular la presion en un punto determinado partiendo del punto cuya presion es conocida e ir adicionando o substrayendo el termino $\rho g h$ en la direccion al punto de presion desconocida. Por ejemplo si tenemos los fluidos de la figura~\ref{mano3} y queremos calcular la presion en el punto 1, empezamos desde la presion conocida $P_{atm}$ y vamos avanzando hasta el punto 1, lo cual da: 

% Fig 3.21 Cengel
\begin{figure}[h]
\centering
\includegraphics[width=10cm]{mano3}
\caption{Tres tipos de fluido en reposo de diferente $\rho$ ubicados unos sobre el otro.}
\label{mano3}
\end{figure}

$$
P_{atm} + \rho_1 g h_1 + \rho_2 g h_2 + \rho_3 g h_3 = P_1
$$
En el caso en que los tres fluidos tuvieran la misma densidad, la ecuacion anterior quedaria: $P_{atm}+\rho g(h_1 + h_2 + h_3) = P_1$

Los manometros son utilizado para medir los cambios presion (generalmente debido a valvulas o accesorios) entre dos secciones de una tuberia con flujo a presion.  Dicho manometro se conecta entre dos secciones de una tuberia (ver figura~\ref{mano4} que transporta liquido o gas cuya densidad es $\rho_1$. La densidad del liquido en el manometro $\rho_2$ debe ser mayor que $\rho_1$ y ambos fluidos deben ser inmisibles. La diferencia de presion $P_1 - P_2$ puede ser calculada iniciando en el punto 1 y moviendose a lo largo del manometro adicionando o restando $\rho g h$ hasta alcanzar el punto 2, lo cual quedaria:

% Fig 3.22 Cengel
\begin{figure}[h]
\centering
\includegraphics[width=10cm]{mano4}
\caption{Medicion de la diferencia de presion en un tuberia con un manometro.}
\label{mano4}
\end{figure}

$$
P_1 + \rho_1 g (a+h) - \rho_2 g h - \rho_1 ga = P_2
$$
Note que los puntos a una distancia $a$ tienen una misma presion por estar al mismo nivel en el manometro, simplificando:
$$
P_1 - P_2 = (\rho_1 - \rho_2)gh
$$

Si el fluido que fluye a lo largo de la tuberia es gas, $\rho_1 \ll \rho_2$ por lo que la ecuacion anterior se convierte en $P_1-P_2 \cong \rho_2 g h$.


\subsection{Otros medidores de presion}
\begin{itemize}
\item Tubo de Bourdon
\item Transductor de presion
\end{itemize}

\section{Estatica de fluidos}
La \textbf{estatica de fluidos} trabaja con fluidos en \textbf{reposo} y es conocida como \textbf{hidroestatica} cuando el fluido es un liquido y \textbf{aeroestatica} cuando el fluido es un gas. En un fluido en reposo no se deforma porque no existen esfuerzos cortantes tangenciales entre capas. El only esfuerzo actuante sobre un fluido en reposo es el \textbf{esfuerzo normal} o la presion la cual varia con el peso del fluido que a su vez es funci\'on de la gravedad. La estatica de fluidos es importante para el dise\~no de objetos y estructuras flotantes o subermegidas como presas, submarinos y tanques de almacenamiento de agua.
\subsection{Fuerza hidroest\'atica sobre superficies planas sumergidas}
\label{hipl}
Una placa sumergida (e.g. compuerta de una presa, pared de un tanque) esta sujeta a una presion ejercida por el fluido sobre su superficie. En una superficie plana, las fuerzas hidroestaticas actuan paralelas y es importante determinar su magnitud y su punto de aplicacion (\textbf{centro de presi\'on}) sobre la superficie. En algunos casos, la superficie esta parcialmente suemergica y mientras que un lado esta en contacto con un fluido, en el otro esta en contacto con al atmosfera por lo tanto la $P_{atm}$ se anula y la presion resultante es la presion hidroestatica $P_{gage}=\rho g h$ como se muestra en la figura~\ref{hes1}.

% Fig 3.27 Cengel
\begin{figure}[h]
\centering
\includegraphics[width=10cm]{hes1}
\caption{Fuerzas hidroestaticas sobre una superficie plana parcialmente sumergida.}
\label{hes1}
\end{figure}

Un caso mas general es el ilustrado en la figura~\ref{hes2} en donde tenemos una placa irregular sumergida e inclinada un angulo $\theta$ con respecto a la horizontal. La presion absoluta ejercida sobre cualquier punto sobre la placa es:
% Fig 3.28 Cengel
\begin{figure}[h]
\centering
\includegraphics[width=10cm]{hes2}
\caption{Fuerza hidroestatica sobre una placa inclinada y sumergida.}
\label{hes2}
\end{figure}

$$
P=P_0 + \rho g h = P_0 + \rho g y \sin \theta
$$
en donde $P_0$ es la presion arriba de el fluido (= $P_{atm}$ si el fluido esta abierto a la atmosfera), $y$ es la distancia a lo largo del mismo eje desde el origen $O$ y $h$ es la distancia vertical desde la superficie del agua hasta la placa en donde $h=y \sin \theta$. 

La fuerza hidroestatica resultante $F_R$ que actua sobre la placa es:
$$
F_R = \int_A P dA = \int_A (P_0 + \rho g y \sin \theta) dA = P_0 A + \rho g \sin \theta \int_A y dA
$$
en donde $dA$ es el diferencial de area sobre la la placa. Como el \texttt{primer momento de area} $\int_A y dA$ se relaciona con la coordenada $y$ del \textbf{centroide} $y_C = \frac{1}{a} \int_A y dA$ de la placa, reemplazando en la ecuacion anterior:
\begin{equation}
F_R = (P_0 + \rho g y_C \sin \theta)A = (P_0 + \rho g h_C)A = P_C A = P_{avg} A
\label{fr}
\end{equation}
en donde $P_C$ es la presion sobre el centroide de la placa la cual es equivalente a la presion promedio $P_{avg}$, y $h_C$ es la distance vertical desde la superficie hasta el centroide.

Cuando $P_0 = P_{atm}$, esta puede ser ignorada en el calculo de $F_R$ ya que actua en ambas caras de la placa. Cuando esto no es asi, la fuerza adicional ejercida debido a $P_0$ se calcula adicionando $h_{equiv}=P_0 /\rho g$ a $h_C$, lo cual asume la presencia de una capa adicional de liquido en la superficie. 

La linea de accion de $F_R$ no pasa por el centroide de la placa; esta pasa por el \textbf{centro de presion}. Para determinar la posicion del centro de presion, estimamos el momento de $F_R$ con respecto al eje $x$:
$$
y_p F_R = \int_A y P dA = \int_A y(P_0 \rho g y \sin \theta)dA = P_0 \int_A y dA + \rho g \sin \theta \int_A y^2 dA
$$
o
\begin{equation}
y_p F_R = P_0 y_C A + \rho g \sin \theta I_{xx,O}
\label{ypfr}
\end{equation}
donde $y_p$ es la distancia al centro de presion desde el eje $x$, y $I_{xx,O} = \int_A y^2 dA$ es el \texttt{segundo momento de area o de inercia} con respecto al eje $x$. Por el teorema de \texttt{ejes parelelos} el $I_{xx,O}$ puede ser calculado con respecto al eje $x$ que pasa por por centroide como:
\begin{equation}
I_{xx,O} = I_{xx,C}+y^2_C A
\label{ixx}
\end{equation}
donde $I_{xx,C}$ es el segundo momento de area con respecto al eje $x$ que pasa por el centroide y $y_C$ es la distancia entre los dos ejes paralelos. Reemplazando las ecuaciones~\ref{fr} y ~\ref{ixx} en la ecuacion~\ref{ypfr}:
$$
y_p (P_0 + \rho g y_C \sin \theta)A = P_0 y_C A + \rho g \sin \theta (I_{xx,C}+y^2_C A)
$$
despejando para $y_p$:
\begin{flalign*}
y_p  &= \frac{P_0 y_C A + \rho g \sin \theta (I_{xx,C}+y^2_C A)}{(P_0 + \rho g y_C \sin \theta)A} \\
&= \frac{P_0 y_C A + \rho g \sin \theta I_{xx,C}+\rho g \sin \theta y^2_C A}{(P_0 + \rho g y_C \sin \theta)A} \\
&= \frac{y_C A (P_0 +\rho g \sin \theta y_C) + \rho g \sin \theta I_{xx,C}}{(P_0 + \rho g y_C \sin \theta)A} \\
&= y_C + \frac{\rho g \sin \theta I_{xx,C}}{(P_0 + \rho g y_C \sin \theta)A}
\end{flalign*}

\begin{equation}
y_p = y_C + \frac{I_{xx,C}}{ \left[ y_C + \frac{P_0}{\rho g \sin \theta} \right]A }
\label{yp}
\end{equation}

Cuando $P_0=P_{atm}$ y es ignorada, $y_p$ es:
\begin{equation}
y_p = y_C + \frac{I_{xx,C}}{y_C A} 
\label{yp1}
\end{equation}

Por lo tanto la distancia vertical desde la superficie hasta el centro de presion es: $h_p = y_p \sin \theta$.

Valores de $I_{xx,C}$ para diferentes areas comunes son mostrados en la figura~\ref{centro}. Para areas simetricas con respecto al eje $y$, el centro de presion esta debajo del centroide y sobre el eje $y$.
% Fig 3.31 Cengel
\begin{figure}[h]
\centering
\includegraphics[width=10cm]{centro}
\caption{El centroide y el momento de inercia con respecto al centroide para figuras geometricas comunes.}
\label{centro}
\end{figure}


\subsection{Fuerza hidroest\'atica sobre superficies curvas sumergidas}
\label{hicu}
En superficies curvas sumergidas, la fuerza hidroestatica cambia de direccion dependiendo de la posicion sobre la superficie. Para facilitar el calculo, es necesario estimar las componentes horizontal $F_H$ y vertical $F_V$ de $F_R$. Si consideramos la superficie curva de la figura~\ref{cur1}, el calculo de $F_H$ y $F_V$ se hace calculando las fuerzas actuantes sobre la proyeccion horizontal y vertical de la superficie curva. Tenemos entonces que:

% Fig 3.36 Cengel
\begin{figure}[h]
\centering
\includegraphics[width=10cm]{cur1}
\caption{Determinaci\'on de la fuerza hidroestatica actuante sobre una superficie curva sumergida.}
\label{cur1}
\end{figure}

$$
F_H=F_x
$$
$$
F_V = F_y \pm W
$$
donde $F_x$ se calcula como la resultante de la fuerza hidroestatica sobre una superficie plana vertical (ver seccion~\ref{hipl}), $F_y$ es la fuerza en $y$ ejercida sobre el fluido (e.g. fluido superior y aire) y $W = \rho g V$ es el peso del volumen ($V$) de fluido. Note que $F_y$ y $W$ se suman si actuan en la misma direccion y se restan si van en direcciones opuestas. De lo anterior:
$$
F_R=\sqrt{F^2_H + F^2_V}
$$
y el angulo que hace $F_R$ con la horizontal es $\alpha = \arctan \frac{F_V}{F_H}$.


\subsection{Fuerzas hidroestaticas en diferentes fluidos}
Las ecuaciones vistas en las secciones~\ref{hipl} y ~\ref{hicu} son validas para fluidos de densidad uniforme. Si el fluido esta conformado por capas de fluidos de diferentes densidades (ver figura~\ref{mden}) la distribucion lineal de presiones cambia en cada capa. Sin embargo las ecuaciones vistas anteriormente pueden ser aplicadas a cada capa $i$ y las $F_{R,i}$ pueden ser adicionadas para calcular la $F_R$:
% Fig 3.39 Cengel
\begin{figure}[h]
\centering
\includegraphics[width=10cm]{mden}
\caption{Fuerzas hidroestaticas sobre placa sumergida en dos fluidos.}
\label{mden}
\end{figure}


$$
F_R = \sum F_{R,i} = \sum P_{C,i}A_i
$$
donde $P_{C,i} =  P_0 + \rho_i h h_{C,i}$ es la presion en el centroide de la porcion de la superficie en el fluido $i$, y $A_i$ es el area de la superficie en ese fluido. El centro de presion de $F_R$ puede ser encontrado calculando la suma de los momentos de cada fuerza $F_{R,i}$ con respecto a un punto determinado (e.g. la superficie de agua.).

\section{Flotaci\'on y estabilidad}
Es comun sentir como un cuerpo determinado pesa menos cuando este esta en un fluido; dicha sensacion se explica debido a que el fluido ejerce una fuerza vertical hacia arriba que tiende a levantar el objeto la cual es denominada \textbf{fuerza de flotaci\'on}, $F_B$. La fuerza de flotaci\'on es causada por el incremento de presion con la profundidad dentro del fluido. Para su analisis, consideremos la placa de la figura~\ref{flota} de espesor $h$ y area $A$ sumergida en un liquido de densidad $\rho_f$ y que es paralela a la superficie libre. El balance de fuerzas sobre la placa, con base en presiones manometricas es:
% Fig 3.41 Cengel
\begin{figure}[h]
\centering
\includegraphics[width=10cm]{flota}
\caption{Fuerzas actuantes sobre una placa sumergida paralela a la superficie.}
\label{flota}
\end{figure}

\begin{equation}
F_B=F_{bottom} - F_{top} = \rho_f g (s+h)A - \rho_f gs A = \rho_f ghA = \rho_f g V
\label{FB}
\end{equation}
donde $V=hA$ es el volumen de la placa. Teniendo en cuenta que $\rho_f g V$ es el peso del liquido cuyo volumen es igual al volumen de la placa, tenemos que \emph{la fuerza de flotaci\'on actuante sobre una placa es igual al peso del liquido desplazado por la placa}. Note que si el fluido es de densidad constante, la fuerza de flotacion es independiente de la profundidad $h$ y  de la densidad de la placa. En terminos generales y para cualquier objeto se afirma que \emph{la fuerza de flotaci\'on actuante sobre un cuerpo de densidad uniforme inmerso en un fluido es igual al peso del fluido desplazado por el cuerpo, y que actual vertical a travez del centroide del volumen desplazado}, esto es conocido como el \textbf{principio de Arquimides} gracias al matematico Griego Arquimides. 

Para objetos flotantes, el peso del cuerpo debe ser igual a la fuerza de flotacion, la cual es el peso del fluido cuyo volumen es igual a el volumen de la porci\'on sumergida del cuerpo flotante. Esto es:
\begin{equation}
F_B = W \rightarrow \rho_f g V_{sub} = \rho_{avg,body}gV_{total} \rightarrow \frac{V_{sub}}{V_{total}} = \frac{\rho_{avg,body}}{\rho_f}
\label{FB2}
\end{equation}
De la ecuaci\'on~\ref{FB2} tenemos que la fraccion de volumen del objeto flotante es igual a la relaci\'on entre la densidad promedio del cuerpo y la densidad del fluido. Analizando la ecuacion~\ref{FB2} (ver figura~\ref{flota2}), un objeto inmerso en un fluido: 1) permanece en reposo dentro del fluido cuando $\rho = \rho_f$, 2) se hunde y cae hasta el fondo dentro de el fluido cuando $\rho > \rho_f$ y 3) flota cuando $\rho < \rho_f$.

% Fig 3.43 Cengel
\begin{figure}[h]
\centering
\includegraphics[width=10cm]{flota2}
\caption{ Diferentes estados de un cuerpo inmerso en un fluido dependiendo de su densidad ($\rho$) y de la del fluido ($\rho_f$).}
\label{flota2}
\end{figure}

Como es sabido, $F_B$ es proporcional a $\rho_f$. En el caso de los gases, la densidad generalmente es mucho menor que la densidad de un liquido. Sin embargo, las fuerzas de flotacion son notorias en el ascenso de aire caliente en ambientes frios, la circulacion de aire en la atmosfera y el ascenso de globos y de bombas con helio. 

\subsection{Estabilidad de cuerpos sumergidos y flotantes}
Una importante aplicacion del concepto de flotacion es el analisis de la estabilidad de un cuerpo flotante o sumergido, lo cual es fundamental en el disen\~o de embarcaciones y submarinos. Existen tres casos que describen la estabilidad de un cuerpo: 1) Un cuerpo permanece \textbf{estable} si cualquier peque\~na perturbacion genera un movimiento que es contrarestado por una fuerza que hace que el cuerpo retorne a su posicion inicial, 2) un cuerpo es \textbf{neutralmente estable} si despues de ser perturbado cambia su posicion inicial, y 3) un cuerpo es \textbf{inestable} si al ser perturbado este no retorna a su posicion inicial y continua en movimiento. Estos tres estados son aplicables a objetos flotantes o sumergidos. Por ejemplo, un cuerpo sumergido o flotante en equilibrio estatico cuyo peso es balanceado por la fuerza de flotacion permanece estable en la \textbf{direccion vertical}.

La \textbf{estabilidad de rotacion} de un cuerpo cuerpo sumergido depende de la localizacion relativa de el \textbf{centro de gravedad} $G$ de el cuerpo y el centro de flotacion $B$, el cual es el centroide del volumen desplazado. Un cuerpo sumergido permanece estable si su fondo es mas pesado lo que hace que $G$ este por debajo de $B$. Esta es la razon por la cual en el diseno de submarinos, los motores estan en la parte mas baja con el fin de localizar el mayor peso posible en las partes bajas del submarino. Lo mismo occurre con los globos, el cual es un cuerpo inmerso en aire, cuyo mayor peso es la canasta que transporta a las personas esta en la parte baja mientras el aire caliente y liviano se hubica en la parte alta. Si $G$ esta por encima de $B$, el cuerpo sumergido es inestable y cualquier perfturbacion hace que el cuerpo se voltee. En el caso en el que $G$ y $B$ coinciden, el cuerpo sumergido es nuetralmente estable, lo cual es el caso de cuerpos cuya densidad es uniforme (ver figura~\ref{rota}).

% Fig 3.49 Cengel
\begin{figure}[h]
\centering
\includegraphics[width=10cm]{rota}
\caption{Estabilidad de un cuerpo sumergido.}
\label{rota}
\end{figure}

Si en un cuerpo $G$ no esta verticalmente alineado con $B$ no es apropiado hablar de estabilidad ya que el cuerpo no esta en un estado de equilibrio y tratara de rotar o moverse por si solo para alcanzarsu estado estable (ver figura~\ref{rota2}). El momemento restaurador que actua en contra de las manecillas del reloj, tratara de rotar el cuerpo en la misma direccion con el fin de alinear $G$ y $B$ verticalmente.
% Fig 3.50 Cengel
\begin{figure}[h]
\centering
\includegraphics[width=10cm]{rota2}
\caption{Cuerpo sumergido en desequilibrio en donde $G$ no esta alineado verticalmente con $B$.}
\label{rota2}
\end{figure}



La estabilidad de rotaci\'on es similar para cuerpos flotantes. Sin embargo, a diferencia de un cuerpo sumergido, un cuerpo flotante puede permanecer estable incluso cuando $G$ esta arriba de $B$ (ver figura~\ref{rota3}). Esto se logra gracias a que $B$ se desplaza hacia $B'$ generando un momento restaurador entre las dos fuerzas para retornar el objeto a su posicion  original. 

Una medida de la estabilidad de un cuerpo flotante es la \textbf{altura metacentrica} $GM$, la cual es la distancia entre $G$ y el \emph{metacentro} $M$ (punto de interseccion entre la linea de accion de las fuerza flotante antes y despues de la rotacion). Con base en la posicion de $M$, un cuerpo flotante es estable si $M$ esta arriba de $G$ por lo que $GM$ es positiva y es inestable si $M$ esta abajo de $G$ por lo que $GM$ es negativa. Si es inestable, el peso y la fuerza de flotacion actuando sobre el cuerpo inclinado genera un momento de volcamiento en lugar de un momento restaurador. Entre mas grande es $GM$, mas estable es el cuerpo flotante. Note que un bote puede inclinarse hasta cierto angulo maximo sin volcarse, sin embargo, si dicho angulo es excedido el bote se volteara y se hundira.  
% Fig 3.51 Cengel
\begin{figure}[h]
\centering
\includegraphics[width=10cm]{rota3}
\caption{Estabilidad de un cuerpo flotante.}
\label{rota3}
\end{figure}

%El mismo principio para calcular fuerzas hidroestaticas sobre superficies puede ser aplicado para calcular fuerzas netas de presion en cuerpos completamente sumergidos o cuerpos en flotaci\'on. Las dos leyes de flotaci\'on descubiertas por \texttt{Arquimides} son:
%\begin{enumerate}
%\item Un cuerpo inmerso en un fluido experimenta una fuerza vertical flotante igual al peso del fluido desplazado por el cuerpo.
%\item Un cuerpo flotante desplaza su propio peso en el fluido en el cual este flota.
%\end{enumerate}

\section{Fluidos en moviento de cuerpo rigido}



\end{document}

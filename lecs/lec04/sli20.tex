\documentclass [xcolor=svgnames, t] {beamer} 
\usepackage[utf8]{inputenc}
\usepackage{booktabs, comment} 
\usepackage[absolute, overlay]{textpos} 
\usepackage{pgfpages}
\usepackage[font=footnotesize]{caption}
\useoutertheme{infolines} 

\AtBeginSection[]{
  \begin{frame}
  \vfill
  \centering
  \begin{beamercolorbox}[sep=8pt,center,shadow=true,rounded=true]{title}
    \usebeamerfont{title}\insertsectionhead\par%
  \end{beamercolorbox}
  \vfill
  \end{frame}
}


%\definecolor{brownbrown}{RGB}{56, 28, 0}
%\definecolor{brownred}{RGB}{228, 0, 43}

%\setbeamercolor{title in head/foot}{bg=brownred, fg=brownbrown}
%\setbeamercolor{author in head/foot}{bg=myuniversity}
\setbeamertemplate{page number in head/foot}{}
\usepackage{csquotes}


\usepackage{amsmath}
\usepackage[makeroom]{cancel}
\usepackage[absolute,overlay]{textpos}
\usepackage{tcolorbox}

%\usepackage{textpos}

\usepackage{tikz}

\usepackage{media9} 

\usetheme{Madrid}
%\definecolor{myuniversity}{RGB}{56, 28, 0}
%\usecolortheme[named=myuniversity]{structure}



\title[Viscosidad]{Clase No.20: Cinem\'atica de los fluidos}
\subtitle{Propiedades cinem\'aticas de los fluidos}
\institute[]{Departamento de Ingenier\'ia Civil y Agr\'icola\\ Facultad de Ingenier\'ia  \\Universidad Nacional de Colombia - Sede Bogot\'a}
\titlegraphic{\includegraphics[height=2.0cm]{escudoUnal.png}}
\author[LAM]{Luis Alejandro Morales \\ \href{https://lamhydro.github.io}{https://lamhydro.github.io}}


%\institute[]{Department of Earth, Environmental, and Planetary Sciences  \\Brown University}
\date{\today}


\addtobeamertemplate{navigation symbols}{}{%
    \usebeamerfont{footline}%
    \usebeamercolor[fg]{footline}%
    \hspace{1em}%
    \insertframenumber/\inserttotalframenumber
}

\begin{document}
\begin{frame}
\maketitle
\end{frame}


%%%%%%%%%%%%%%%%%%%%%%%%%%%%
\logo{\vspace{-0.2cm}\includegraphics[height=0.8cm]{escudoUnal.png}~%
}
%%%%%%%%%%%%%%%%%%%%%%%%%%



\begin{frame}
\frametitle{Table of Contents}
\tableofcontents
\end{frame}

\section{Introduccion}
\begin{frame}{Introduccion}
\begin{block}{Cinem\'atica de los fluidos}
Estudia el movimiento de las particulas de fluido sin considerar las fuerzas que actuan sobre las mismas; caracteriza dicho movimiento en funcion del espacio y del tiempo.
\end{block}
\end{frame} 

\section{Definiciones}
\begin{frame}{Definiciones}
Algunas definiciones importantes del analisis vectorial son:
\begin{block}{Escalar}
Se define por la magnitud que acquiere la magnitud f\'isica. Ejemplos: temperatura, presion, densidad
\end{block}
\end{frame}

\begin{frame}{Vector}
\begin{block}{Vector: Definicion}
Es una cantidad que tiene magnitud, direcci\'on y sentido. Ejemplos: velocidad, aceleraci\'on, fuerza. Un campo de velocidades para un $t=t_1$, puede estar expresado como:
$$
\vec{U}(x,y,z)=u_x \vec{i} + u_y \vec{j} + u_z \vec{k}
$$ 
donde $u_x$, $u_y$ y $u_z$ son las componentes en $x$, $y$ y $z$, respectivamente, del vector $\vec{U}$ en donde cada componente es una $f(x,y,z)$. $\vec{i}$ $\vec{j}$ $\vec{z}$ son los vectores unitarios (magnitud 1) para $x$, $y$ y $z$, respectivamente. 
Entre dos vectores $\vec{A}$ y $\vec{B}$ es posible efectuar dos tipos de productos:
\begin{enumerate}
\item \emph{Producto scalar o punto}
$$
\vec{A}\dot \vec{B} = ABcos \theta
$$
donde $\theta$ es el angulo formado por los dos vectores. De acuerdo con esto $\vec{i} \cdot \vec{i} = \vec{j} \cdot \vec{j}= \vec{k} \cdot \vec{k} = 1$, mientras, por ejemplo, $\vec{i} \cdot \vec{j} = \vec{j} \cdot \vec{k} =\vec{i} \cdot \vec{k} = 0$.
\item \emph{Producto vectorial o cruz}
$$
\vec{A} x \vec{B} = \vec{n}AB sen \theta
$$
donde $\vec{n}$ es el vector unitario $ixj=k$,$jxk=i$,$k x i = j$, $ixk=-j$ 
\end{enumerate}
\end{block}
\end{frame}


\begin{frame}{Vector: Operadores vectoriales}
\begin{block}{Operador $\nabla$}
Vector simbolico que se aplica a cantidades \emph{escalares} y \emph{vectoriales}, se define como:
\begin{equation}
\nabla = \frac{\partial }{\partial x}\vec{i} + \frac{\partial }{\partial y}\vec{j} + \frac{\partial }{\partial z}\vec{k}
\label{nb}
\end{equation}
\end{block}
\begin{block}{Gradiente de una funci\'on}
Si el operador $\nabla$ se aplica a una funcion escalar $\phi$, se obtiene un vector gradiente definido como:
\begin{equation}
\nabla \phi = \frac{\partial \phi}{\partial x}\vec{i} + \frac{\partial \phi}{\partial y}\vec{j} + \frac{\partial \phi}{\partial z}\vec{k}
\label{phi}
\end{equation}
\end{block}
\end{frame}

\begin{frame}{Vector: Operadores vectoriales}
\begin{block}{Divergencia}
El producto punto entre el operador $\nabla$ y un vector $\vec{U}$, se obtiene un escalar conocido como la divergencia de $\vec{U}$, que se expresa como:
\begin{equation}
\nabla \cdot \vec{U} = \left[ \frac{\partial }{\partial x}\vec{i} + \frac{\partial }{\partial y}\vec{j} + \frac{\partial }{\partial z}\vec{k} \right] [u\vec{i}+v\vec{j}+w\vec{k}] = \frac{\partial u}{\partial x}+\frac{\partial v}{\partial y}+\frac{\partial w}{\partial z}
\label{di1}
\end{equation}
Note que $\nabla \cdot \vec{U} \ne \vec{U} \cdot \nabla$, entonces:
\begin{equation*}
\vec{U} \cdot \nabla  = [u\vec{i}+v\vec{j}+w\vec{k}] \left[ \frac{\partial }{\partial x}\vec{i} + \frac{\partial }{\partial y}\vec{j} + \frac{\partial }{\partial z}\vec{k} \right]  = u\frac{\partial}{\partial x}+v\frac{\partial}{\partial y}+w\frac{\partial}{\partial z}
\end{equation*}
\end{block}
\end{frame}

\begin{frame}{Vector: Operadores vectoriales}
\begin{block}{Rotacional}
Es el producto cruz entre el operador $\nabla$ y un vector $\vec{U}$, se obtiene un vector conocido como el rotacional de $\vec{U}$, que se expresa como:
\begin{equation}
\nabla \cdot \vec{U} = \left[ \frac{\partial }{\partial x}\vec{i} + \frac{\partial }{\partial y}\vec{j} + \frac{\partial }{\partial z}\vec{k} \right] x [u\vec{i}+v\vec{j}+w\vec{k}] = XXXX
\label{rot}
\end{equation}
\end{block}
\end{frame}

\begin{frame}{Vector: Operadores vectoriales}
\vspace{-0.5cm}
\footnotesize
\begin{block}{Laplaciano}
Se define como:
\begin{equation}
\nabla \cdot \nabla = \nabla^2 = \left[ \frac{\partial^2 }{\partial x^2} + \frac{\partial^2}{\partial y^2} + \frac{\partial^2}{\partial z^2} \right]
\label{lap}
\end{equation}
Si $\nabla^2$ se aplica sobre una funcion escalar $\phi$, se obtiene:
\begin{equation}
\nabla^2 \phi = \left[ \frac{\partial^2 \phi}{\partial x^2} + \frac{\partial^2 \phi}{\partial y^2} + \frac{\partial^2 \phi}{\partial z^2} \right]
\label{lap2}
\end{equation}
Si $\nabla^2$ se aplica sobre un vector  $\vec{U}$ (e.g. velocidad), se obtiene:
$$
\nabla^2 \vec{U} = \left[ \frac{\partial^2}{\partial x^2} + \frac{\partial^2}{\partial y^2} + \frac{\partial^2}{\partial z^2} \right][u\vec{i}+v\vec{j}+w\vec{k}]
$$
\begin{align}
\nabla^2 \vec{U} = \left[ \frac{\partial^2 u}{\partial x^2} + \frac{\partial^2 u}{\partial y^2} + \frac{\partial^2 u}{\partial z^2} \right]\vec{i} &+ \left[ \frac{\partial^2 v}{\partial x^2} + \frac{\partial^2 v}{\partial y^2} + \frac{\partial^2 v}{\partial z^2} \right]\vec{j} + \left[ \frac{\partial^2 w}{\partial x^2} + \frac{\partial^2 w}{\partial y^2} + \frac{\partial^2 w}{\partial z^2} \right]\vec{k}\\
&\nabla^2 \vec{U} =\vec{i} \nabla^2 u + \vec{j} \nabla^2 v + \vec{k} \nabla^2 w
\label{lap3}
\end{align}
\end{block}
\end{frame}


\begin{frame}{Campo escalar}
\begin{block}{Campo escalar}
Un escalar $\phi$ es funcion de las coordenadas espaciales $x$, $y$ y $z$ o del vector de posici\'on $\vec{r}$ del punto $P(x,y,z)$; $\vec{r}$ une al origen del sistema de referencia con  $P$. Es posible entonces escribir:
$$
\phi = \phi(x,y,z) = \phi(P)=\phi(\vec{r})
$$
Por lo tanto un campo escalar queda definido si para cada punto $P$ existe un \'unico valor de $\phi$ exigiendose que la funcion $\phi(x,y,z)$ sea continua y derivable en el espacio. El espacio geom\'etrico para el cual multiple puntos $(x,y,z)$ tienen el mismo valor de $\phi$ se denomina \alert{superficie equipotencial}. Si $\phi=f(x,y)$, el lugar geometrico de todos los punto con igual valor $\phi$ es una \alert{linea equipotencial}. La presion en un fluido es un ejemplo de un campo escalar.
\end{block}
\end{frame}

\begin{frame}{Campo vectorial y potencial}
\vspace{-0.5cm}
\begin{block}{Campo vectorial}
Si un vector $\vec{A}$ es funci\'on de su posici\'on $(x,y,z)$ en el espacio, se puede escribir que:
$$
\vec{A} = \vec{A}(x,y,z) = \vec{A}(P) = \vec{A}(\vec{r}) = a(x,y,z)\vec{i} + b(x,y,z)\vec{j} + c(x,y,z)\vec{k}
$$
Por lo tanto un \alert{campo vectorial} esta definido si existe un valor de $\vec{A}$ en capa punto $P(x,y,z)$. La velocidad de un fluido $\vec{U}$ es un ejemplo de un campo vectorial.
\end{block}
\begin{block}{Campo potencial}
Un campo potencial es aquel en el que:
$$
\vec{U} = \nabla \phi = \frac{\partial \phi}{\partial x}\vec{i} + \frac{\partial \phi}{\partial y}\vec{j} + \frac{\partial \phi}{\partial z}\vec{k}
$$
en donde $\vec{U}$ es la velocidad y $\phi$ es una funci\'on escalar.
\end{block}
\end{frame}

\begin{frame}{Concepto de flujo de un campo vectorial y de circulaci\'on de un vector}
\vspace{-0.5cm}
\begin{block}{Concepto de flujo de un campo vectorial}
Supongase que existe un campo vetorial $\vec{A}=\vec{A}(\vec{r})$ y una superficie $S$ delimitada por una linea $L$. Un diferencial $dS$  esta definida por el vector unitario $\vec{n}$ perpendicular a $dS$. Asi el flujo del vector $A$ a trav\'es de la superficie $S$ es:
$$
\psi = \int_S \vec{A} \cdot \vec{n}dS = \int_S \vec{A} \cdot dS 
$$
\end{block}
\vspace{-0.23cm}
\begin{block}{Concepto de circulaci\'on de un vector}
La circulaci\'on de un vector $\vec{A}$ a lo largo de la linea $L$ est\'a definida como:
$$
\Gamma = \int_L \vec{A} \cdot d\vec{L}
$$
Si $\vec{A}$ representa un campo de fuerza $\vec{F}$, $\Gamma$ representa el trabajo mec\'anico realizado por $\vec{F}$.
\end{block}
\end{frame}

\section{Teorema de Gauss de la divergencia}
\begin{frame}{Teorema de Gauss de la divergencia}
\begin{block}{Teorema de Gauss de la divergencia}

\end{block}
\end{frame}


\end{document}

